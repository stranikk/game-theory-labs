\documentclass[11pt]{article}
    \usepackage[russian]{babel}

    \usepackage[breakable]{tcolorbox}
    \usepackage{parskip} % Stop auto-indenting (to mimic markdown behaviour)
    
    \usepackage{iftex}
    \ifPDFTeX
    	\usepackage[T1]{fontenc}
    	\usepackage{mathpazo}
    \else
    	\usepackage{fontspec}
    \fi

    % Basic figure setup, for now with no caption control since it's done
    % automatically by Pandoc (which extracts ![](path) syntax from Markdown).
    \usepackage{graphicx}
    % Maintain compatibility with old templates. Remove in nbconvert 6.0
    \let\Oldincludegraphics\includegraphics
    % Ensure that by default, figures have no caption (until we provide a
    % proper Figure object with a Caption API and a way to capture that
    % in the conversion process - todo).
    \usepackage{caption}
    \DeclareCaptionFormat{nocaption}{}
    \captionsetup{format=nocaption,aboveskip=0pt,belowskip=0pt}

    \usepackage[Export]{adjustbox} % Used to constrain images to a maximum size
    \adjustboxset{max size={0.9\linewidth}{0.9\paperheight}}
    \usepackage{float}
    \floatplacement{figure}{H} % forces figures to be placed at the correct location
    \usepackage{xcolor} % Allow colors to be defined
    \usepackage{enumerate} % Needed for markdown enumerations to work
    \usepackage{geometry} % Used to adjust the document margins
    \usepackage{amsmath} % Equations
    \usepackage{amssymb} % Equations
    \usepackage{textcomp} % defines textquotesingle
    % Hack from http://tex.stackexchange.com/a/47451/13684:
    \AtBeginDocument{%
        \def\PYZsq{\textquotesingle}% Upright quotes in Pygmentized code
    }
    \usepackage{upquote} % Upright quotes for verbatim code
    \usepackage{eurosym} % defines \euro
    \usepackage[mathletters]{ucs} % Extended unicode (utf-8) support
    \usepackage{fancyvrb} % verbatim replacement that allows latex
    \usepackage{grffile} % extends the file name processing of package graphics 
                         % to support a larger range
    \makeatletter % fix for grffile with XeLaTeX
    \def\Gread@@xetex#1{%
      \IfFileExists{"\Gin@base".bb}%
      {\Gread@eps{\Gin@base.bb}}%
      {\Gread@@xetex@aux#1}%
    }
    \makeatother

    % The hyperref package gives us a pdf with properly built
    % internal navigation ('pdf bookmarks' for the table of contents,
    % internal cross-reference links, web links for URLs, etc.)
    \usepackage{hyperref}
    % The default LaTeX title has an obnoxious amount of whitespace. By default,
    % titling removes some of it. It also provides customization options.
    \usepackage{titling}
    \usepackage{longtable} % longtable support required by pandoc >1.10
    \usepackage{booktabs}  % table support for pandoc > 1.12.2
    \usepackage[inline]{enumitem} % IRkernel/repr support (it uses the enumerate* environment)
    \usepackage[normalem]{ulem} % ulem is needed to support strikethroughs (\sout)
                                % normalem makes italics be italics, not underlines
    \usepackage{mathrsfs}
    

    
    % Colors for the hyperref package
    \definecolor{urlcolor}{rgb}{0,.145,.698}
    \definecolor{linkcolor}{rgb}{.71,0.21,0.01}
    \definecolor{citecolor}{rgb}{.12,.54,.11}

    % ANSI colors
    \definecolor{ansi-black}{HTML}{3E424D}
    \definecolor{ansi-black-intense}{HTML}{282C36}
    \definecolor{ansi-red}{HTML}{E75C58}
    \definecolor{ansi-red-intense}{HTML}{B22B31}
    \definecolor{ansi-green}{HTML}{00A250}
    \definecolor{ansi-green-intense}{HTML}{007427}
    \definecolor{ansi-yellow}{HTML}{DDB62B}
    \definecolor{ansi-yellow-intense}{HTML}{B27D12}
    \definecolor{ansi-blue}{HTML}{208FFB}
    \definecolor{ansi-blue-intense}{HTML}{0065CA}
    \definecolor{ansi-magenta}{HTML}{D160C4}
    \definecolor{ansi-magenta-intense}{HTML}{A03196}
    \definecolor{ansi-cyan}{HTML}{60C6C8}
    \definecolor{ansi-cyan-intense}{HTML}{258F8F}
    \definecolor{ansi-white}{HTML}{C5C1B4}
    \definecolor{ansi-white-intense}{HTML}{A1A6B2}
    \definecolor{ansi-default-inverse-fg}{HTML}{FFFFFF}
    \definecolor{ansi-default-inverse-bg}{HTML}{000000}

    % commands and environments needed by pandoc snippets
    % extracted from the output of `pandoc -s`
    \providecommand{\tightlist}{%
      \setlength{\itemsep}{0pt}\setlength{\parskip}{0pt}}
    \DefineVerbatimEnvironment{Highlighting}{Verbatim}{commandchars=\\\{\}}
    % Add ',fontsize=\small' for more characters per line
    \newenvironment{Shaded}{}{}
    \newcommand{\KeywordTok}[1]{\textcolor[rgb]{0.00,0.44,0.13}{\textbf{{#1}}}}
    \newcommand{\DataTypeTok}[1]{\textcolor[rgb]{0.56,0.13,0.00}{{#1}}}
    \newcommand{\DecValTok}[1]{\textcolor[rgb]{0.25,0.63,0.44}{{#1}}}
    \newcommand{\BaseNTok}[1]{\textcolor[rgb]{0.25,0.63,0.44}{{#1}}}
    \newcommand{\FloatTok}[1]{\textcolor[rgb]{0.25,0.63,0.44}{{#1}}}
    \newcommand{\CharTok}[1]{\textcolor[rgb]{0.25,0.44,0.63}{{#1}}}
    \newcommand{\StringTok}[1]{\textcolor[rgb]{0.25,0.44,0.63}{{#1}}}
    \newcommand{\CommentTok}[1]{\textcolor[rgb]{0.38,0.63,0.69}{\textit{{#1}}}}
    \newcommand{\OtherTok}[1]{\textcolor[rgb]{0.00,0.44,0.13}{{#1}}}
    \newcommand{\AlertTok}[1]{\textcolor[rgb]{1.00,0.00,0.00}{\textbf{{#1}}}}
    \newcommand{\FunctionTok}[1]{\textcolor[rgb]{0.02,0.16,0.49}{{#1}}}
    \newcommand{\RegionMarkerTok}[1]{{#1}}
    \newcommand{\ErrorTok}[1]{\textcolor[rgb]{1.00,0.00,0.00}{\textbf{{#1}}}}
    \newcommand{\NormalTok}[1]{{#1}}
    
    % Additional commands for more recent versions of Pandoc
    \newcommand{\ConstantTok}[1]{\textcolor[rgb]{0.53,0.00,0.00}{{#1}}}
    \newcommand{\SpecialCharTok}[1]{\textcolor[rgb]{0.25,0.44,0.63}{{#1}}}
    \newcommand{\VerbatimStringTok}[1]{\textcolor[rgb]{0.25,0.44,0.63}{{#1}}}
    \newcommand{\SpecialStringTok}[1]{\textcolor[rgb]{0.73,0.40,0.53}{{#1}}}
    \newcommand{\ImportTok}[1]{{#1}}
    \newcommand{\DocumentationTok}[1]{\textcolor[rgb]{0.73,0.13,0.13}{\textit{{#1}}}}
    \newcommand{\AnnotationTok}[1]{\textcolor[rgb]{0.38,0.63,0.69}{\textbf{\textit{{#1}}}}}
    \newcommand{\CommentVarTok}[1]{\textcolor[rgb]{0.38,0.63,0.69}{\textbf{\textit{{#1}}}}}
    \newcommand{\VariableTok}[1]{\textcolor[rgb]{0.10,0.09,0.49}{{#1}}}
    \newcommand{\ControlFlowTok}[1]{\textcolor[rgb]{0.00,0.44,0.13}{\textbf{{#1}}}}
    \newcommand{\OperatorTok}[1]{\textcolor[rgb]{0.40,0.40,0.40}{{#1}}}
    \newcommand{\BuiltInTok}[1]{{#1}}
    \newcommand{\ExtensionTok}[1]{{#1}}
    \newcommand{\PreprocessorTok}[1]{\textcolor[rgb]{0.74,0.48,0.00}{{#1}}}
    \newcommand{\AttributeTok}[1]{\textcolor[rgb]{0.49,0.56,0.16}{{#1}}}
    \newcommand{\InformationTok}[1]{\textcolor[rgb]{0.38,0.63,0.69}{\textbf{\textit{{#1}}}}}
    \newcommand{\WarningTok}[1]{\textcolor[rgb]{0.38,0.63,0.69}{\textbf{\textit{{#1}}}}}
    
    
    % Define a nice break command that doesn't care if a line doesn't already
    % exist.
    \def\br{\hspace*{\fill} \\* }
    % Math Jax compatibility definitions
    \def\gt{>}
    \def\lt{<}
    \let\Oldtex\TeX
    \let\Oldlatex\LaTeX
    \renewcommand{\TeX}{\textrm{\Oldtex}}
    \renewcommand{\LaTeX}{\textrm{\Oldlatex}}
    % Document parameters
    % Document title
    \title{lab1}
    
    
    
    
    
% Pygments definitions
\makeatletter
\def\PY@reset{\let\PY@it=\relax \let\PY@bf=\relax%
    \let\PY@ul=\relax \let\PY@tc=\relax%
    \let\PY@bc=\relax \let\PY@ff=\relax}
\def\PY@tok#1{\csname PY@tok@#1\endcsname}
\def\PY@toks#1+{\ifx\relax#1\empty\else%
    \PY@tok{#1}\expandafter\PY@toks\fi}
\def\PY@do#1{\PY@bc{\PY@tc{\PY@ul{%
    \PY@it{\PY@bf{\PY@ff{#1}}}}}}}
\def\PY#1#2{\PY@reset\PY@toks#1+\relax+\PY@do{#2}}

\expandafter\def\csname PY@tok@w\endcsname{\def\PY@tc##1{\textcolor[rgb]{0.73,0.73,0.73}{##1}}}
\expandafter\def\csname PY@tok@c\endcsname{\let\PY@it=\textit\def\PY@tc##1{\textcolor[rgb]{0.25,0.50,0.50}{##1}}}
\expandafter\def\csname PY@tok@cp\endcsname{\def\PY@tc##1{\textcolor[rgb]{0.74,0.48,0.00}{##1}}}
\expandafter\def\csname PY@tok@k\endcsname{\let\PY@bf=\textbf\def\PY@tc##1{\textcolor[rgb]{0.00,0.50,0.00}{##1}}}
\expandafter\def\csname PY@tok@kp\endcsname{\def\PY@tc##1{\textcolor[rgb]{0.00,0.50,0.00}{##1}}}
\expandafter\def\csname PY@tok@kt\endcsname{\def\PY@tc##1{\textcolor[rgb]{0.69,0.00,0.25}{##1}}}
\expandafter\def\csname PY@tok@o\endcsname{\def\PY@tc##1{\textcolor[rgb]{0.40,0.40,0.40}{##1}}}
\expandafter\def\csname PY@tok@ow\endcsname{\let\PY@bf=\textbf\def\PY@tc##1{\textcolor[rgb]{0.67,0.13,1.00}{##1}}}
\expandafter\def\csname PY@tok@nb\endcsname{\def\PY@tc##1{\textcolor[rgb]{0.00,0.50,0.00}{##1}}}
\expandafter\def\csname PY@tok@nf\endcsname{\def\PY@tc##1{\textcolor[rgb]{0.00,0.00,1.00}{##1}}}
\expandafter\def\csname PY@tok@nc\endcsname{\let\PY@bf=\textbf\def\PY@tc##1{\textcolor[rgb]{0.00,0.00,1.00}{##1}}}
\expandafter\def\csname PY@tok@nn\endcsname{\let\PY@bf=\textbf\def\PY@tc##1{\textcolor[rgb]{0.00,0.00,1.00}{##1}}}
\expandafter\def\csname PY@tok@ne\endcsname{\let\PY@bf=\textbf\def\PY@tc##1{\textcolor[rgb]{0.82,0.25,0.23}{##1}}}
\expandafter\def\csname PY@tok@nv\endcsname{\def\PY@tc##1{\textcolor[rgb]{0.10,0.09,0.49}{##1}}}
\expandafter\def\csname PY@tok@no\endcsname{\def\PY@tc##1{\textcolor[rgb]{0.53,0.00,0.00}{##1}}}
\expandafter\def\csname PY@tok@nl\endcsname{\def\PY@tc##1{\textcolor[rgb]{0.63,0.63,0.00}{##1}}}
\expandafter\def\csname PY@tok@ni\endcsname{\let\PY@bf=\textbf\def\PY@tc##1{\textcolor[rgb]{0.60,0.60,0.60}{##1}}}
\expandafter\def\csname PY@tok@na\endcsname{\def\PY@tc##1{\textcolor[rgb]{0.49,0.56,0.16}{##1}}}
\expandafter\def\csname PY@tok@nt\endcsname{\let\PY@bf=\textbf\def\PY@tc##1{\textcolor[rgb]{0.00,0.50,0.00}{##1}}}
\expandafter\def\csname PY@tok@nd\endcsname{\def\PY@tc##1{\textcolor[rgb]{0.67,0.13,1.00}{##1}}}
\expandafter\def\csname PY@tok@s\endcsname{\def\PY@tc##1{\textcolor[rgb]{0.73,0.13,0.13}{##1}}}
\expandafter\def\csname PY@tok@sd\endcsname{\let\PY@it=\textit\def\PY@tc##1{\textcolor[rgb]{0.73,0.13,0.13}{##1}}}
\expandafter\def\csname PY@tok@si\endcsname{\let\PY@bf=\textbf\def\PY@tc##1{\textcolor[rgb]{0.73,0.40,0.53}{##1}}}
\expandafter\def\csname PY@tok@se\endcsname{\let\PY@bf=\textbf\def\PY@tc##1{\textcolor[rgb]{0.73,0.40,0.13}{##1}}}
\expandafter\def\csname PY@tok@sr\endcsname{\def\PY@tc##1{\textcolor[rgb]{0.73,0.40,0.53}{##1}}}
\expandafter\def\csname PY@tok@ss\endcsname{\def\PY@tc##1{\textcolor[rgb]{0.10,0.09,0.49}{##1}}}
\expandafter\def\csname PY@tok@sx\endcsname{\def\PY@tc##1{\textcolor[rgb]{0.00,0.50,0.00}{##1}}}
\expandafter\def\csname PY@tok@m\endcsname{\def\PY@tc##1{\textcolor[rgb]{0.40,0.40,0.40}{##1}}}
\expandafter\def\csname PY@tok@gh\endcsname{\let\PY@bf=\textbf\def\PY@tc##1{\textcolor[rgb]{0.00,0.00,0.50}{##1}}}
\expandafter\def\csname PY@tok@gu\endcsname{\let\PY@bf=\textbf\def\PY@tc##1{\textcolor[rgb]{0.50,0.00,0.50}{##1}}}
\expandafter\def\csname PY@tok@gd\endcsname{\def\PY@tc##1{\textcolor[rgb]{0.63,0.00,0.00}{##1}}}
\expandafter\def\csname PY@tok@gi\endcsname{\def\PY@tc##1{\textcolor[rgb]{0.00,0.63,0.00}{##1}}}
\expandafter\def\csname PY@tok@gr\endcsname{\def\PY@tc##1{\textcolor[rgb]{1.00,0.00,0.00}{##1}}}
\expandafter\def\csname PY@tok@ge\endcsname{\let\PY@it=\textit}
\expandafter\def\csname PY@tok@gs\endcsname{\let\PY@bf=\textbf}
\expandafter\def\csname PY@tok@gp\endcsname{\let\PY@bf=\textbf\def\PY@tc##1{\textcolor[rgb]{0.00,0.00,0.50}{##1}}}
\expandafter\def\csname PY@tok@go\endcsname{\def\PY@tc##1{\textcolor[rgb]{0.53,0.53,0.53}{##1}}}
\expandafter\def\csname PY@tok@gt\endcsname{\def\PY@tc##1{\textcolor[rgb]{0.00,0.27,0.87}{##1}}}
\expandafter\def\csname PY@tok@err\endcsname{\def\PY@bc##1{\setlength{\fboxsep}{0pt}\fcolorbox[rgb]{1.00,0.00,0.00}{1,1,1}{\strut ##1}}}
\expandafter\def\csname PY@tok@kc\endcsname{\let\PY@bf=\textbf\def\PY@tc##1{\textcolor[rgb]{0.00,0.50,0.00}{##1}}}
\expandafter\def\csname PY@tok@kd\endcsname{\let\PY@bf=\textbf\def\PY@tc##1{\textcolor[rgb]{0.00,0.50,0.00}{##1}}}
\expandafter\def\csname PY@tok@kn\endcsname{\let\PY@bf=\textbf\def\PY@tc##1{\textcolor[rgb]{0.00,0.50,0.00}{##1}}}
\expandafter\def\csname PY@tok@kr\endcsname{\let\PY@bf=\textbf\def\PY@tc##1{\textcolor[rgb]{0.00,0.50,0.00}{##1}}}
\expandafter\def\csname PY@tok@bp\endcsname{\def\PY@tc##1{\textcolor[rgb]{0.00,0.50,0.00}{##1}}}
\expandafter\def\csname PY@tok@fm\endcsname{\def\PY@tc##1{\textcolor[rgb]{0.00,0.00,1.00}{##1}}}
\expandafter\def\csname PY@tok@vc\endcsname{\def\PY@tc##1{\textcolor[rgb]{0.10,0.09,0.49}{##1}}}
\expandafter\def\csname PY@tok@vg\endcsname{\def\PY@tc##1{\textcolor[rgb]{0.10,0.09,0.49}{##1}}}
\expandafter\def\csname PY@tok@vi\endcsname{\def\PY@tc##1{\textcolor[rgb]{0.10,0.09,0.49}{##1}}}
\expandafter\def\csname PY@tok@vm\endcsname{\def\PY@tc##1{\textcolor[rgb]{0.10,0.09,0.49}{##1}}}
\expandafter\def\csname PY@tok@sa\endcsname{\def\PY@tc##1{\textcolor[rgb]{0.73,0.13,0.13}{##1}}}
\expandafter\def\csname PY@tok@sb\endcsname{\def\PY@tc##1{\textcolor[rgb]{0.73,0.13,0.13}{##1}}}
\expandafter\def\csname PY@tok@sc\endcsname{\def\PY@tc##1{\textcolor[rgb]{0.73,0.13,0.13}{##1}}}
\expandafter\def\csname PY@tok@dl\endcsname{\def\PY@tc##1{\textcolor[rgb]{0.73,0.13,0.13}{##1}}}
\expandafter\def\csname PY@tok@s2\endcsname{\def\PY@tc##1{\textcolor[rgb]{0.73,0.13,0.13}{##1}}}
\expandafter\def\csname PY@tok@sh\endcsname{\def\PY@tc##1{\textcolor[rgb]{0.73,0.13,0.13}{##1}}}
\expandafter\def\csname PY@tok@s1\endcsname{\def\PY@tc##1{\textcolor[rgb]{0.73,0.13,0.13}{##1}}}
\expandafter\def\csname PY@tok@mb\endcsname{\def\PY@tc##1{\textcolor[rgb]{0.40,0.40,0.40}{##1}}}
\expandafter\def\csname PY@tok@mf\endcsname{\def\PY@tc##1{\textcolor[rgb]{0.40,0.40,0.40}{##1}}}
\expandafter\def\csname PY@tok@mh\endcsname{\def\PY@tc##1{\textcolor[rgb]{0.40,0.40,0.40}{##1}}}
\expandafter\def\csname PY@tok@mi\endcsname{\def\PY@tc##1{\textcolor[rgb]{0.40,0.40,0.40}{##1}}}
\expandafter\def\csname PY@tok@il\endcsname{\def\PY@tc##1{\textcolor[rgb]{0.40,0.40,0.40}{##1}}}
\expandafter\def\csname PY@tok@mo\endcsname{\def\PY@tc##1{\textcolor[rgb]{0.40,0.40,0.40}{##1}}}
\expandafter\def\csname PY@tok@ch\endcsname{\let\PY@it=\textit\def\PY@tc##1{\textcolor[rgb]{0.25,0.50,0.50}{##1}}}
\expandafter\def\csname PY@tok@cm\endcsname{\let\PY@it=\textit\def\PY@tc##1{\textcolor[rgb]{0.25,0.50,0.50}{##1}}}
\expandafter\def\csname PY@tok@cpf\endcsname{\let\PY@it=\textit\def\PY@tc##1{\textcolor[rgb]{0.25,0.50,0.50}{##1}}}
\expandafter\def\csname PY@tok@c1\endcsname{\let\PY@it=\textit\def\PY@tc##1{\textcolor[rgb]{0.25,0.50,0.50}{##1}}}
\expandafter\def\csname PY@tok@cs\endcsname{\let\PY@it=\textit\def\PY@tc##1{\textcolor[rgb]{0.25,0.50,0.50}{##1}}}

\def\PYZbs{\char`\\}
\def\PYZus{\char`\_}
\def\PYZob{\char`\{}
\def\PYZcb{\char`\}}
\def\PYZca{\char`\^}
\def\PYZam{\char`\&}
\def\PYZlt{\char`\<}
\def\PYZgt{\char`\>}
\def\PYZsh{\char`\#}
\def\PYZpc{\char`\%}
\def\PYZdl{\char`\$}
\def\PYZhy{\char`\-}
\def\PYZsq{\char`\'}
\def\PYZdq{\char`\"}
\def\PYZti{\char`\~}
% for compatibility with earlier versions
\def\PYZat{@}
\def\PYZlb{[}
\def\PYZrb{]}
\makeatother


    % For linebreaks inside Verbatim environment from package fancyvrb. 
    \makeatletter
        \newbox\Wrappedcontinuationbox 
        \newbox\Wrappedvisiblespacebox 
        \newcommand*\Wrappedvisiblespace {\textcolor{red}{\textvisiblespace}} 
        \newcommand*\Wrappedcontinuationsymbol {\textcolor{red}{\llap{\tiny$\m@th\hookrightarrow$}}} 
        \newcommand*\Wrappedcontinuationindent {3ex } 
        \newcommand*\Wrappedafterbreak {\kern\Wrappedcontinuationindent\copy\Wrappedcontinuationbox} 
        % Take advantage of the already applied Pygments mark-up to insert 
        % potential linebreaks for TeX processing. 
        %        {, <, #, %, $, ' and ": go to next line. 
        %        _, }, ^, &, >, - and ~: stay at end of broken line. 
        % Use of \textquotesingle for straight quote. 
        \newcommand*\Wrappedbreaksatspecials {% 
            \def\PYGZus{\discretionary{\char`\_}{\Wrappedafterbreak}{\char`\_}}% 
            \def\PYGZob{\discretionary{}{\Wrappedafterbreak\char`\{}{\char`\{}}% 
            \def\PYGZcb{\discretionary{\char`\}}{\Wrappedafterbreak}{\char`\}}}% 
            \def\PYGZca{\discretionary{\char`\^}{\Wrappedafterbreak}{\char`\^}}% 
            \def\PYGZam{\discretionary{\char`\&}{\Wrappedafterbreak}{\char`\&}}% 
            \def\PYGZlt{\discretionary{}{\Wrappedafterbreak\char`\<}{\char`\<}}% 
            \def\PYGZgt{\discretionary{\char`\>}{\Wrappedafterbreak}{\char`\>}}% 
            \def\PYGZsh{\discretionary{}{\Wrappedafterbreak\char`\#}{\char`\#}}% 
            \def\PYGZpc{\discretionary{}{\Wrappedafterbreak\char`\%}{\char`\%}}% 
            \def\PYGZdl{\discretionary{}{\Wrappedafterbreak\char`\$}{\char`\$}}% 
            \def\PYGZhy{\discretionary{\char`\-}{\Wrappedafterbreak}{\char`\-}}% 
            \def\PYGZsq{\discretionary{}{\Wrappedafterbreak\textquotesingle}{\textquotesingle}}% 
            \def\PYGZdq{\discretionary{}{\Wrappedafterbreak\char`\"}{\char`\"}}% 
            \def\PYGZti{\discretionary{\char`\~}{\Wrappedafterbreak}{\char`\~}}% 
        } 
        % Some characters . , ; ? ! / are not pygmentized. 
        % This macro makes them "active" and they will insert potential linebreaks 
        \newcommand*\Wrappedbreaksatpunct {% 
            \lccode`\~`\.\lowercase{\def~}{\discretionary{\hbox{\char`\.}}{\Wrappedafterbreak}{\hbox{\char`\.}}}% 
            \lccode`\~`\,\lowercase{\def~}{\discretionary{\hbox{\char`\,}}{\Wrappedafterbreak}{\hbox{\char`\,}}}% 
            \lccode`\~`\;\lowercase{\def~}{\discretionary{\hbox{\char`\;}}{\Wrappedafterbreak}{\hbox{\char`\;}}}% 
            \lccode`\~`\:\lowercase{\def~}{\discretionary{\hbox{\char`\:}}{\Wrappedafterbreak}{\hbox{\char`\:}}}% 
            \lccode`\~`\?\lowercase{\def~}{\discretionary{\hbox{\char`\?}}{\Wrappedafterbreak}{\hbox{\char`\?}}}% 
            \lccode`\~`\!\lowercase{\def~}{\discretionary{\hbox{\char`\!}}{\Wrappedafterbreak}{\hbox{\char`\!}}}% 
            \lccode`\~`\/\lowercase{\def~}{\discretionary{\hbox{\char`\/}}{\Wrappedafterbreak}{\hbox{\char`\/}}}% 
            \catcode`\.\active
            \catcode`\,\active 
            \catcode`\;\active
            \catcode`\:\active
            \catcode`\?\active
            \catcode`\!\active
            \catcode`\/\active 
            \lccode`\~`\~ 	
        }
    \makeatother

    \let\OriginalVerbatim=\Verbatim
    \makeatletter
    \renewcommand{\Verbatim}[1][1]{%
        %\parskip\z@skip
        \sbox\Wrappedcontinuationbox {\Wrappedcontinuationsymbol}%
        \sbox\Wrappedvisiblespacebox {\FV@SetupFont\Wrappedvisiblespace}%
        \def\FancyVerbFormatLine ##1{\hsize\linewidth
            \vtop{\raggedright\hyphenpenalty\z@\exhyphenpenalty\z@
                \doublehyphendemerits\z@\finalhyphendemerits\z@
                \strut ##1\strut}%
        }%
        % If the linebreak is at a space, the latter will be displayed as visible
        % space at end of first line, and a continuation symbol starts next line.
        % Stretch/shrink are however usually zero for typewriter font.
        \def\FV@Space {%
            \nobreak\hskip\z@ plus\fontdimen3\font minus\fontdimen4\font
            \discretionary{\copy\Wrappedvisiblespacebox}{\Wrappedafterbreak}
            {\kern\fontdimen2\font}%
        }%
        
        % Allow breaks at special characters using \PYG... macros.
        \Wrappedbreaksatspecials
        % Breaks at punctuation characters . , ; ? ! and / need catcode=\active 	
        \OriginalVerbatim[#1,codes*=\Wrappedbreaksatpunct]%
    }
    \makeatother

    % Exact colors from NB
    \definecolor{incolor}{HTML}{303F9F}
    \definecolor{outcolor}{HTML}{D84315}
    \definecolor{cellborder}{HTML}{CFCFCF}
    \definecolor{cellbackground}{HTML}{F7F7F7}
    
    % prompt
    \makeatletter
    \newcommand{\boxspacing}{\kern\kvtcb@left@rule\kern\kvtcb@boxsep}
    \makeatother
    \newcommand{\prompt}[4]{
        \ttfamily\llap{{\color{#2}[#3]:\hspace{3pt}#4}}\vspace{-\baselineskip}
    }
    

    
    % Prevent overflowing lines due to hard-to-break entities
    \sloppy 
    % Setup hyperref package
    \hypersetup{
      breaklinks=true,  % so long urls are correctly broken across lines
      colorlinks=true,
      urlcolor=urlcolor,
      linkcolor=linkcolor,
      citecolor=citecolor,
      }
    % Slightly bigger margins than the latex defaults
    
    \geometry{verbose,tmargin=1in,bmargin=1in,lmargin=1in,rmargin=1in}
    
    

\begin{document}
   
    \hypertarget{ux43bux430ux431ux43eux440ux430ux442ux43eux440ux43dux430ux44f-ux440ux430ux431ux43eux442ux430-1}{%
\section{Лабораторная работа
№1}\label{ux43bux430ux431ux43eux440ux430ux442ux43eux440ux43dux430ux44f-ux440ux430ux431ux43eux442ux430-1}}

Цель работы - изучить постановку антагонистической игры двух лиц в
нормальной форме; найти решение игры за обоих игроков в смешанных
стратегиях (стратегическую седловую точку).

\hypertarget{ux43fux43eux441ux442ux430ux43dux43eux432ux43aux430-ux437ux430ux434ux430ux447ux438-ux438-ux43cux435ux442ux43eux434ux438ux447ux435ux441ux43aux438ux435-ux443ux43aux430ux437ux430ux43dux438ux44f}{%
\subsection{Постановка задачи и методические
указания}\label{ux43fux43eux441ux442ux430ux43dux43eux432ux43aux430-ux437ux430ux434ux430ux447ux438-ux438-ux43cux435ux442ux43eux434ux438ux447ux435ux441ux43aux438ux435-ux443ux43aux430ux437ux430ux43dux438ux44f}}

Для игры, заданной матрицей \(c_{ij}\), требуется найти оптимальные
смешанные стратегии обоих игроков, сведя матричную игру к задаче ЛП
(прямой для одного игрока и двойственной для другого).

Задачи ЛП следует решать симплекс-методом, приводя начальные,
промежуточные и конечные симплекс-таблицы. По окончании алгоритма
полученные решения необходимо проверить на допустимость.

    \hypertarget{ux445ux43eux434-ux440ux430ux431ux43eux442ux44b}{%
\subsection{Ход
работы}\label{ux445ux43eux434-ux440ux430ux431ux43eux442ux44b}}

Зададим матрицу стратегий:

    \begin{tcolorbox}[breakable, size=fbox, boxrule=1pt, pad at break*=1mm,colback=cellbackground, colframe=cellborder]
\prompt{In}{incolor}{1}{\boxspacing}
\begin{Verbatim}[commandchars=\\\{\}]
\PY{k+kn}{from} \PY{n+nn}{IPython}\PY{n+nn}{.}\PY{n+nn}{core}\PY{n+nn}{.}\PY{n+nn}{display} \PY{k+kn}{import} \PY{n}{display}\PY{p}{,} \PY{n}{HTML}\PY{p}{,} \PY{n}{Latex}

\PY{n}{C} \PY{o}{=} \PY{n}{matrix}\PY{p}{(}\PY{n}{SR}\PY{p}{,} \PY{l+m+mi}{4}\PY{p}{,} \PY{l+m+mi}{5}\PY{p}{,} \PY{p}{[}\PY{l+m+mi}{8}\PY{p}{,} \PY{l+m+mi}{1}\PY{p}{,} \PY{l+m+mi}{17}\PY{p}{,} \PY{l+m+mi}{8}\PY{p}{,} \PY{l+m+mi}{1}\PY{p}{,} \PY{l+m+mi}{12}\PY{p}{,} \PY{l+m+mi}{6}\PY{p}{,} \PY{l+m+mi}{11}\PY{p}{,} \PY{l+m+mi}{10}\PY{p}{,} \PY{l+m+mi}{16}\PY{p}{,} \PY{l+m+mi}{4}\PY{p}{,} \PY{l+m+mi}{19}\PY{p}{,} \PY{l+m+mi}{11}\PY{p}{,} \PY{l+m+mi}{15}\PY{p}{,} \PY{l+m+mi}{2}\PY{p}{,} \PY{l+m+mi}{17}\PY{p}{,} \PY{l+m+mi}{19}\PY{p}{,} \PY{l+m+mi}{6}\PY{p}{,} \PY{l+m+mi}{17}\PY{p}{,} \PY{l+m+mi}{16}\PY{p}{]}\PY{p}{)}
\PY{n}{C}
\end{Verbatim}
\end{tcolorbox}

            \begin{tcolorbox}[breakable, size=fbox, boxrule=.5pt, pad at break*=1mm, opacityfill=0]
\prompt{Out}{outcolor}{1}{\boxspacing}
\begin{Verbatim}[commandchars=\\\{\}]
[ 8  1 17  8  1]
[12  6 11 10 16]
[ 4 19 11 15  2]
[17 19  6 17 16]
\end{Verbatim}
\end{tcolorbox}
        
    Сформулируем задачу линейного программирования для решения
симплекс-методом для игрока \(A\):

    \begin{tcolorbox}[breakable, size=fbox, boxrule=1pt, pad at break*=1mm,colback=cellbackground, colframe=cellborder]
\prompt{In}{incolor}{2}{\boxspacing}
\begin{Verbatim}[commandchars=\\\{\}]
\PY{n}{A} \PY{o}{=} \PY{n}{C}\PY{o}{.}\PY{n}{transpose}\PY{p}{(}\PY{p}{)}
\PY{n}{b} \PY{o}{=} \PY{p}{[}\PY{l+m+mi}{1}\PY{p}{]} \PY{o}{*} \PY{n}{A}\PY{o}{.}\PY{n}{nrows}\PY{p}{(}\PY{p}{)}
\PY{n}{c} \PY{o}{=} \PY{p}{[}\PY{l+m+mi}{1}\PY{p}{]} \PY{o}{*} \PY{n}{A}\PY{o}{.}\PY{n}{ncols}\PY{p}{(}\PY{p}{)}

\PY{n}{P} \PY{o}{=} \PY{n}{InteractiveLPProblem}\PY{p}{(}\PY{n}{A}\PY{p}{,} \PY{n}{b}\PY{p}{,} \PY{n}{c}\PY{p}{,} \PY{n}{constraint\PYZus{}type}\PY{o}{=}\PY{p}{[}\PY{l+s+s2}{\PYZdq{}}\PY{l+s+s2}{\PYZgt{}=}\PY{l+s+s2}{\PYZdq{}}\PY{p}{]} \PY{o}{*} \PY{n+nb}{len}\PY{p}{(}\PY{n}{b}\PY{p}{)}\PY{p}{,} \PY{n}{problem\PYZus{}type}\PY{o}{=}\PY{l+s+s2}{\PYZdq{}}\PY{l+s+s2}{min}\PY{l+s+s2}{\PYZdq{}}\PY{p}{,} \PY{n}{variable\PYZus{}type}\PY{o}{=}\PY{l+s+s1}{\PYZsq{}}\PY{l+s+s1}{\PYZgt{}=}\PY{l+s+s1}{\PYZsq{}}\PY{p}{)}
\end{Verbatim}
\end{tcolorbox}

    Решим задачу ЛП симплекс-методом:

    \begin{tcolorbox}[breakable, size=fbox, boxrule=1pt, pad at break*=1mm,colback=cellbackground, colframe=cellborder]
\prompt{In}{incolor}{3}{\boxspacing}
\begin{Verbatim}[commandchars=\\\{\}]
\PY{n}{P} \PY{o}{=} \PY{n}{P}\PY{o}{.}\PY{n}{standard\PYZus{}form}\PY{p}{(}\PY{p}{)}
\PY{n}{Latex}\PY{p}{(}\PY{n}{P}\PY{o}{.}\PY{n}{run\PYZus{}simplex\PYZus{}method}\PY{p}{(}\PY{p}{)}\PY{p}{)}
\end{Verbatim}
\end{tcolorbox}
 
            
\prompt{Out}{outcolor}{3}{}
    
    \begin{equation*}
\renewcommand{\arraystretch}{1.5} %notruncate
\begin{array}{|rcrcrcrcrcr|}
\hline
x_{5} \mspace{-6mu}&\mspace{-6mu} = \mspace{-6mu}&\mspace{-6mu} -1 \mspace{-6mu}&\mspace{-6mu} + \mspace{-6mu}&\mspace{-6mu} 8 x_{1} \mspace{-6mu}&\mspace{-6mu} + \mspace{-6mu}&\mspace{-6mu} 12 x_{2} \mspace{-6mu}&\mspace{-6mu} + \mspace{-6mu}&\mspace{-6mu} 4 x_{3} \mspace{-6mu}&\mspace{-6mu} + \mspace{-6mu}&\mspace{-6mu} 17 x_{4}\\
x_{6} \mspace{-6mu}&\mspace{-6mu} = \mspace{-6mu}&\mspace{-6mu} -1 \mspace{-6mu}&\mspace{-6mu} + \mspace{-6mu}&\mspace{-6mu} x_{1} \mspace{-6mu}&\mspace{-6mu} + \mspace{-6mu}&\mspace{-6mu} 6 x_{2} \mspace{-6mu}&\mspace{-6mu} + \mspace{-6mu}&\mspace{-6mu} 19 x_{3} \mspace{-6mu}&\mspace{-6mu} + \mspace{-6mu}&\mspace{-6mu} 19 x_{4}\\
x_{7} \mspace{-6mu}&\mspace{-6mu} = \mspace{-6mu}&\mspace{-6mu} -1 \mspace{-6mu}&\mspace{-6mu} + \mspace{-6mu}&\mspace{-6mu} 17 x_{1} \mspace{-6mu}&\mspace{-6mu} + \mspace{-6mu}&\mspace{-6mu} 11 x_{2} \mspace{-6mu}&\mspace{-6mu} + \mspace{-6mu}&\mspace{-6mu} 11 x_{3} \mspace{-6mu}&\mspace{-6mu} + \mspace{-6mu}&\mspace{-6mu} 6 x_{4}\\
x_{8} \mspace{-6mu}&\mspace{-6mu} = \mspace{-6mu}&\mspace{-6mu} -1 \mspace{-6mu}&\mspace{-6mu} + \mspace{-6mu}&\mspace{-6mu} 8 x_{1} \mspace{-6mu}&\mspace{-6mu} + \mspace{-6mu}&\mspace{-6mu} 10 x_{2} \mspace{-6mu}&\mspace{-6mu} + \mspace{-6mu}&\mspace{-6mu} 15 x_{3} \mspace{-6mu}&\mspace{-6mu} + \mspace{-6mu}&\mspace{-6mu} 17 x_{4}\\
x_{9} \mspace{-6mu}&\mspace{-6mu} = \mspace{-6mu}&\mspace{-6mu} -1 \mspace{-6mu}&\mspace{-6mu} + \mspace{-6mu}&\mspace{-6mu} x_{1} \mspace{-6mu}&\mspace{-6mu} + \mspace{-6mu}&\mspace{-6mu} 16 x_{2} \mspace{-6mu}&\mspace{-6mu} + \mspace{-6mu}&\mspace{-6mu} 2 x_{3} \mspace{-6mu}&\mspace{-6mu} + \mspace{-6mu}&\mspace{-6mu} 16 x_{4}\\
\hline
-z \mspace{-6mu}&\mspace{-6mu} = \mspace{-6mu}&\mspace{-6mu} 0 \mspace{-6mu}&\mspace{-6mu} - \mspace{-6mu}&\mspace{-6mu} x_{1} \mspace{-6mu}&\mspace{-6mu} - \mspace{-6mu}&\mspace{-6mu} x_{2} \mspace{-6mu}&\mspace{-6mu} - \mspace{-6mu}&\mspace{-6mu} x_{3} \mspace{-6mu}&\mspace{-6mu} - \mspace{-6mu}&\mspace{-6mu} x_{4}\\
\hline
\end{array}
\end{equation*}
The initial dictionary is infeasible, solving auxiliary problem.
\begin{equation*}
\renewcommand{\arraystretch}{1.5} %notruncate
\begin{array}{|rcrcrcrcrcrcr|}
\hline
\color{red}x_{5} \mspace{-6mu}&\color{red}\mspace{-6mu} = \mspace{-6mu}&\color{red}\mspace{-6mu} -1 \mspace{-6mu}&\color{red}\mspace{-6mu} + \mspace{-6mu}&\color{blue}\mspace{-6mu} x_{0} \mspace{-6mu}&\color{red}\mspace{-6mu} + \mspace{-6mu}&\color{red}\mspace{-6mu} 8 x_{1} \mspace{-6mu}&\color{red}\mspace{-6mu} + \mspace{-6mu}&\color{red}\mspace{-6mu} 12 x_{2} \mspace{-6mu}&\color{red}\mspace{-6mu} + \mspace{-6mu}&\color{red}\mspace{-6mu} 4 x_{3} \mspace{-6mu}&\color{red}\mspace{-6mu} + \mspace{-6mu}&\color{red}\mspace{-6mu} 17 x_{4}\\
x_{6} \mspace{-6mu}&\mspace{-6mu} = \mspace{-6mu}&\mspace{-6mu} -1 \mspace{-6mu}&\mspace{-6mu} + \mspace{-6mu}&\color{green}\mspace{-6mu} x_{0} \mspace{-6mu}&\mspace{-6mu} + \mspace{-6mu}&\mspace{-6mu} x_{1} \mspace{-6mu}&\mspace{-6mu} + \mspace{-6mu}&\mspace{-6mu} 6 x_{2} \mspace{-6mu}&\mspace{-6mu} + \mspace{-6mu}&\mspace{-6mu} 19 x_{3} \mspace{-6mu}&\mspace{-6mu} + \mspace{-6mu}&\mspace{-6mu} 19 x_{4}\\
x_{7} \mspace{-6mu}&\mspace{-6mu} = \mspace{-6mu}&\mspace{-6mu} -1 \mspace{-6mu}&\mspace{-6mu} + \mspace{-6mu}&\color{green}\mspace{-6mu} x_{0} \mspace{-6mu}&\mspace{-6mu} + \mspace{-6mu}&\mspace{-6mu} 17 x_{1} \mspace{-6mu}&\mspace{-6mu} + \mspace{-6mu}&\mspace{-6mu} 11 x_{2} \mspace{-6mu}&\mspace{-6mu} + \mspace{-6mu}&\mspace{-6mu} 11 x_{3} \mspace{-6mu}&\mspace{-6mu} + \mspace{-6mu}&\mspace{-6mu} 6 x_{4}\\
x_{8} \mspace{-6mu}&\mspace{-6mu} = \mspace{-6mu}&\mspace{-6mu} -1 \mspace{-6mu}&\mspace{-6mu} + \mspace{-6mu}&\color{green}\mspace{-6mu} x_{0} \mspace{-6mu}&\mspace{-6mu} + \mspace{-6mu}&\mspace{-6mu} 8 x_{1} \mspace{-6mu}&\mspace{-6mu} + \mspace{-6mu}&\mspace{-6mu} 10 x_{2} \mspace{-6mu}&\mspace{-6mu} + \mspace{-6mu}&\mspace{-6mu} 15 x_{3} \mspace{-6mu}&\mspace{-6mu} + \mspace{-6mu}&\mspace{-6mu} 17 x_{4}\\
x_{9} \mspace{-6mu}&\mspace{-6mu} = \mspace{-6mu}&\mspace{-6mu} -1 \mspace{-6mu}&\mspace{-6mu} + \mspace{-6mu}&\color{green}\mspace{-6mu} x_{0} \mspace{-6mu}&\mspace{-6mu} + \mspace{-6mu}&\mspace{-6mu} x_{1} \mspace{-6mu}&\mspace{-6mu} + \mspace{-6mu}&\mspace{-6mu} 16 x_{2} \mspace{-6mu}&\mspace{-6mu} + \mspace{-6mu}&\mspace{-6mu} 2 x_{3} \mspace{-6mu}&\mspace{-6mu} + \mspace{-6mu}&\mspace{-6mu} 16 x_{4}\\
\hline
w \mspace{-6mu}&\mspace{-6mu} = \mspace{-6mu}&\mspace{-6mu} 0 \mspace{-6mu}&\mspace{-6mu} - \mspace{-6mu}&\color{green}\mspace{-6mu} x_{0} \mspace{-6mu}&\mspace{-6mu}  \mspace{-6mu}&\mspace{-6mu}  \mspace{-6mu}&\mspace{-6mu}  \mspace{-6mu}&\mspace{-6mu}  \mspace{-6mu}&\mspace{-6mu}  \mspace{-6mu}&\mspace{-6mu}  \mspace{-6mu}&\mspace{-6mu}  \mspace{-6mu}&\mspace{-6mu} \\
\hline
\end{array}
\end{equation*}
Entering: $x_{0}$. Leaving: $x_{5}$. 
\begin{equation*}
\renewcommand{\arraystretch}{1.5} %notruncate
\begin{array}{|rcrcrcrcrcrcr|}
\hline
x_{0} \mspace{-6mu}&\mspace{-6mu} = \mspace{-6mu}&\mspace{-6mu} 1 \mspace{-6mu}&\mspace{-6mu} + \mspace{-6mu}&\mspace{-6mu} x_{5} \mspace{-6mu}&\mspace{-6mu} - \mspace{-6mu}&\color{green}\mspace{-6mu} 8 x_{1} \mspace{-6mu}&\mspace{-6mu} - \mspace{-6mu}&\mspace{-6mu} 12 x_{2} \mspace{-6mu}&\mspace{-6mu} - \mspace{-6mu}&\mspace{-6mu} 4 x_{3} \mspace{-6mu}&\mspace{-6mu} - \mspace{-6mu}&\mspace{-6mu} 17 x_{4}\\
\color{red}x_{6} \mspace{-6mu}&\color{red}\mspace{-6mu} = \mspace{-6mu}&\color{red}\mspace{-6mu} 0 \mspace{-6mu}&\color{red}\mspace{-6mu} + \mspace{-6mu}&\color{red}\mspace{-6mu} x_{5} \mspace{-6mu}&\color{red}\mspace{-6mu} - \mspace{-6mu}&\color{blue}\mspace{-6mu} 7 x_{1} \mspace{-6mu}&\color{red}\mspace{-6mu} - \mspace{-6mu}&\color{red}\mspace{-6mu} 6 x_{2} \mspace{-6mu}&\color{red}\mspace{-6mu} + \mspace{-6mu}&\color{red}\mspace{-6mu} 15 x_{3} \mspace{-6mu}&\color{red}\mspace{-6mu} + \mspace{-6mu}&\color{red}\mspace{-6mu} 2 x_{4}\\
x_{7} \mspace{-6mu}&\mspace{-6mu} = \mspace{-6mu}&\mspace{-6mu} 0 \mspace{-6mu}&\mspace{-6mu} + \mspace{-6mu}&\mspace{-6mu} x_{5} \mspace{-6mu}&\mspace{-6mu} + \mspace{-6mu}&\color{green}\mspace{-6mu} 9 x_{1} \mspace{-6mu}&\mspace{-6mu} - \mspace{-6mu}&\mspace{-6mu} x_{2} \mspace{-6mu}&\mspace{-6mu} + \mspace{-6mu}&\mspace{-6mu} 7 x_{3} \mspace{-6mu}&\mspace{-6mu} - \mspace{-6mu}&\mspace{-6mu} 11 x_{4}\\
x_{8} \mspace{-6mu}&\mspace{-6mu} = \mspace{-6mu}&\mspace{-6mu} 0 \mspace{-6mu}&\mspace{-6mu} + \mspace{-6mu}&\mspace{-6mu} x_{5} \mspace{-6mu}&\mspace{-6mu}  \mspace{-6mu}&\color{green}\mspace{-6mu}  \mspace{-6mu}&\mspace{-6mu} - \mspace{-6mu}&\mspace{-6mu} 2 x_{2} \mspace{-6mu}&\mspace{-6mu} + \mspace{-6mu}&\mspace{-6mu} 11 x_{3} \mspace{-6mu}&\mspace{-6mu}  \mspace{-6mu}&\mspace{-6mu} \\
x_{9} \mspace{-6mu}&\mspace{-6mu} = \mspace{-6mu}&\mspace{-6mu} 0 \mspace{-6mu}&\mspace{-6mu} + \mspace{-6mu}&\mspace{-6mu} x_{5} \mspace{-6mu}&\mspace{-6mu} - \mspace{-6mu}&\color{green}\mspace{-6mu} 7 x_{1} \mspace{-6mu}&\mspace{-6mu} + \mspace{-6mu}&\mspace{-6mu} 4 x_{2} \mspace{-6mu}&\mspace{-6mu} - \mspace{-6mu}&\mspace{-6mu} 2 x_{3} \mspace{-6mu}&\mspace{-6mu} - \mspace{-6mu}&\mspace{-6mu} x_{4}\\
\hline
w \mspace{-6mu}&\mspace{-6mu} = \mspace{-6mu}&\mspace{-6mu} -1 \mspace{-6mu}&\mspace{-6mu} - \mspace{-6mu}&\mspace{-6mu} x_{5} \mspace{-6mu}&\mspace{-6mu} + \mspace{-6mu}&\color{green}\mspace{-6mu} 8 x_{1} \mspace{-6mu}&\mspace{-6mu} + \mspace{-6mu}&\mspace{-6mu} 12 x_{2} \mspace{-6mu}&\mspace{-6mu} + \mspace{-6mu}&\mspace{-6mu} 4 x_{3} \mspace{-6mu}&\mspace{-6mu} + \mspace{-6mu}&\mspace{-6mu} 17 x_{4}\\
\hline
\end{array}
\end{equation*}
Entering: $x_{1}$. Leaving: $x_{6}$. 
\begin{equation*}
\renewcommand{\arraystretch}{1.5} %notruncate
\begin{array}{|rcrcrcrcrcrcr|}
\hline
x_{0} \mspace{-6mu}&\mspace{-6mu} = \mspace{-6mu}&\mspace{-6mu} 1 \mspace{-6mu}&\mspace{-6mu} - \mspace{-6mu}&\mspace{-6mu} \frac{1}{7} x_{5} \mspace{-6mu}&\mspace{-6mu} + \mspace{-6mu}&\mspace{-6mu} \frac{8}{7} x_{6} \mspace{-6mu}&\mspace{-6mu} - \mspace{-6mu}&\color{green}\mspace{-6mu} \frac{36}{7} x_{2} \mspace{-6mu}&\mspace{-6mu} - \mspace{-6mu}&\mspace{-6mu} \frac{148}{7} x_{3} \mspace{-6mu}&\mspace{-6mu} - \mspace{-6mu}&\mspace{-6mu} \frac{135}{7} x_{4}\\
\color{red}x_{1} \mspace{-6mu}&\color{red}\mspace{-6mu} = \mspace{-6mu}&\color{red}\mspace{-6mu} 0 \mspace{-6mu}&\color{red}\mspace{-6mu} + \mspace{-6mu}&\color{red}\mspace{-6mu} \frac{1}{7} x_{5} \mspace{-6mu}&\color{red}\mspace{-6mu} - \mspace{-6mu}&\color{red}\mspace{-6mu} \frac{1}{7} x_{6} \mspace{-6mu}&\color{red}\mspace{-6mu} - \mspace{-6mu}&\color{blue}\mspace{-6mu} \frac{6}{7} x_{2} \mspace{-6mu}&\color{red}\mspace{-6mu} + \mspace{-6mu}&\color{red}\mspace{-6mu} \frac{15}{7} x_{3} \mspace{-6mu}&\color{red}\mspace{-6mu} + \mspace{-6mu}&\color{red}\mspace{-6mu} \frac{2}{7} x_{4}\\
x_{7} \mspace{-6mu}&\mspace{-6mu} = \mspace{-6mu}&\mspace{-6mu} 0 \mspace{-6mu}&\mspace{-6mu} + \mspace{-6mu}&\mspace{-6mu} \frac{16}{7} x_{5} \mspace{-6mu}&\mspace{-6mu} - \mspace{-6mu}&\mspace{-6mu} \frac{9}{7} x_{6} \mspace{-6mu}&\mspace{-6mu} - \mspace{-6mu}&\color{green}\mspace{-6mu} \frac{61}{7} x_{2} \mspace{-6mu}&\mspace{-6mu} + \mspace{-6mu}&\mspace{-6mu} \frac{184}{7} x_{3} \mspace{-6mu}&\mspace{-6mu} - \mspace{-6mu}&\mspace{-6mu} \frac{59}{7} x_{4}\\
x_{8} \mspace{-6mu}&\mspace{-6mu} = \mspace{-6mu}&\mspace{-6mu} 0 \mspace{-6mu}&\mspace{-6mu} + \mspace{-6mu}&\mspace{-6mu} x_{5} \mspace{-6mu}&\mspace{-6mu}  \mspace{-6mu}&\mspace{-6mu}  \mspace{-6mu}&\mspace{-6mu} - \mspace{-6mu}&\color{green}\mspace{-6mu} 2 x_{2} \mspace{-6mu}&\mspace{-6mu} + \mspace{-6mu}&\mspace{-6mu} 11 x_{3} \mspace{-6mu}&\mspace{-6mu}  \mspace{-6mu}&\mspace{-6mu} \\
x_{9} \mspace{-6mu}&\mspace{-6mu} = \mspace{-6mu}&\mspace{-6mu} 0 \mspace{-6mu}&\mspace{-6mu}  \mspace{-6mu}&\mspace{-6mu}  \mspace{-6mu}&\mspace{-6mu} + \mspace{-6mu}&\mspace{-6mu} x_{6} \mspace{-6mu}&\mspace{-6mu} + \mspace{-6mu}&\color{green}\mspace{-6mu} 10 x_{2} \mspace{-6mu}&\mspace{-6mu} - \mspace{-6mu}&\mspace{-6mu} 17 x_{3} \mspace{-6mu}&\mspace{-6mu} - \mspace{-6mu}&\mspace{-6mu} 3 x_{4}\\
\hline
w \mspace{-6mu}&\mspace{-6mu} = \mspace{-6mu}&\mspace{-6mu} -1 \mspace{-6mu}&\mspace{-6mu} + \mspace{-6mu}&\mspace{-6mu} \frac{1}{7} x_{5} \mspace{-6mu}&\mspace{-6mu} - \mspace{-6mu}&\mspace{-6mu} \frac{8}{7} x_{6} \mspace{-6mu}&\mspace{-6mu} + \mspace{-6mu}&\color{green}\mspace{-6mu} \frac{36}{7} x_{2} \mspace{-6mu}&\mspace{-6mu} + \mspace{-6mu}&\mspace{-6mu} \frac{148}{7} x_{3} \mspace{-6mu}&\mspace{-6mu} + \mspace{-6mu}&\mspace{-6mu} \frac{135}{7} x_{4}\\
\hline
\end{array}
\end{equation*}
Entering: $x_{2}$. Leaving: $x_{1}$. 
\begin{equation*}
\renewcommand{\arraystretch}{1.5} %notruncate
\begin{array}{|rcrcrcrcrcrcr|}
\hline
\color{red}x_{0} \mspace{-6mu}&\color{red}\mspace{-6mu} = \mspace{-6mu}&\color{red}\mspace{-6mu} 1 \mspace{-6mu}&\color{red}\mspace{-6mu} - \mspace{-6mu}&\color{red}\mspace{-6mu} x_{5} \mspace{-6mu}&\color{red}\mspace{-6mu} + \mspace{-6mu}&\color{red}\mspace{-6mu} 2 x_{6} \mspace{-6mu}&\color{red}\mspace{-6mu} + \mspace{-6mu}&\color{red}\mspace{-6mu} 6 x_{1} \mspace{-6mu}&\color{red}\mspace{-6mu} - \mspace{-6mu}&\color{blue}\mspace{-6mu} 34 x_{3} \mspace{-6mu}&\color{red}\mspace{-6mu} - \mspace{-6mu}&\color{red}\mspace{-6mu} 21 x_{4}\\
x_{2} \mspace{-6mu}&\mspace{-6mu} = \mspace{-6mu}&\mspace{-6mu} 0 \mspace{-6mu}&\mspace{-6mu} + \mspace{-6mu}&\mspace{-6mu} \frac{1}{6} x_{5} \mspace{-6mu}&\mspace{-6mu} - \mspace{-6mu}&\mspace{-6mu} \frac{1}{6} x_{6} \mspace{-6mu}&\mspace{-6mu} - \mspace{-6mu}&\mspace{-6mu} \frac{7}{6} x_{1} \mspace{-6mu}&\mspace{-6mu} + \mspace{-6mu}&\color{green}\mspace{-6mu} \frac{5}{2} x_{3} \mspace{-6mu}&\mspace{-6mu} + \mspace{-6mu}&\mspace{-6mu} \frac{1}{3} x_{4}\\
x_{7} \mspace{-6mu}&\mspace{-6mu} = \mspace{-6mu}&\mspace{-6mu} 0 \mspace{-6mu}&\mspace{-6mu} + \mspace{-6mu}&\mspace{-6mu} \frac{5}{6} x_{5} \mspace{-6mu}&\mspace{-6mu} + \mspace{-6mu}&\mspace{-6mu} \frac{1}{6} x_{6} \mspace{-6mu}&\mspace{-6mu} + \mspace{-6mu}&\mspace{-6mu} \frac{61}{6} x_{1} \mspace{-6mu}&\mspace{-6mu} + \mspace{-6mu}&\color{green}\mspace{-6mu} \frac{9}{2} x_{3} \mspace{-6mu}&\mspace{-6mu} - \mspace{-6mu}&\mspace{-6mu} \frac{34}{3} x_{4}\\
x_{8} \mspace{-6mu}&\mspace{-6mu} = \mspace{-6mu}&\mspace{-6mu} 0 \mspace{-6mu}&\mspace{-6mu} + \mspace{-6mu}&\mspace{-6mu} \frac{2}{3} x_{5} \mspace{-6mu}&\mspace{-6mu} + \mspace{-6mu}&\mspace{-6mu} \frac{1}{3} x_{6} \mspace{-6mu}&\mspace{-6mu} + \mspace{-6mu}&\mspace{-6mu} \frac{7}{3} x_{1} \mspace{-6mu}&\mspace{-6mu} + \mspace{-6mu}&\color{green}\mspace{-6mu} 6 x_{3} \mspace{-6mu}&\mspace{-6mu} - \mspace{-6mu}&\mspace{-6mu} \frac{2}{3} x_{4}\\
x_{9} \mspace{-6mu}&\mspace{-6mu} = \mspace{-6mu}&\mspace{-6mu} 0 \mspace{-6mu}&\mspace{-6mu} + \mspace{-6mu}&\mspace{-6mu} \frac{5}{3} x_{5} \mspace{-6mu}&\mspace{-6mu} - \mspace{-6mu}&\mspace{-6mu} \frac{2}{3} x_{6} \mspace{-6mu}&\mspace{-6mu} - \mspace{-6mu}&\mspace{-6mu} \frac{35}{3} x_{1} \mspace{-6mu}&\mspace{-6mu} + \mspace{-6mu}&\color{green}\mspace{-6mu} 8 x_{3} \mspace{-6mu}&\mspace{-6mu} + \mspace{-6mu}&\mspace{-6mu} \frac{1}{3} x_{4}\\
\hline
w \mspace{-6mu}&\mspace{-6mu} = \mspace{-6mu}&\mspace{-6mu} -1 \mspace{-6mu}&\mspace{-6mu} + \mspace{-6mu}&\mspace{-6mu} x_{5} \mspace{-6mu}&\mspace{-6mu} - \mspace{-6mu}&\mspace{-6mu} 2 x_{6} \mspace{-6mu}&\mspace{-6mu} - \mspace{-6mu}&\mspace{-6mu} 6 x_{1} \mspace{-6mu}&\mspace{-6mu} + \mspace{-6mu}&\color{green}\mspace{-6mu} 34 x_{3} \mspace{-6mu}&\mspace{-6mu} + \mspace{-6mu}&\mspace{-6mu} 21 x_{4}\\
\hline
\end{array}
\end{equation*}
Entering: $x_{3}$. Leaving: $x_{0}$. 
\begin{equation*}
\renewcommand{\arraystretch}{1.5} %notruncate
\begin{array}{|rcrcrcrcrcrcr|}
\hline
x_{3} \mspace{-6mu}&\mspace{-6mu} = \mspace{-6mu}&\mspace{-6mu} \frac{1}{34} \mspace{-6mu}&\mspace{-6mu} - \mspace{-6mu}&\mspace{-6mu} \frac{1}{34} x_{5} \mspace{-6mu}&\mspace{-6mu} + \mspace{-6mu}&\mspace{-6mu} \frac{1}{17} x_{6} \mspace{-6mu}&\mspace{-6mu} + \mspace{-6mu}&\mspace{-6mu} \frac{3}{17} x_{1} \mspace{-6mu}&\mspace{-6mu} - \mspace{-6mu}&\mspace{-6mu} \frac{1}{34} x_{0} \mspace{-6mu}&\mspace{-6mu} - \mspace{-6mu}&\mspace{-6mu} \frac{21}{34} x_{4}\\
x_{2} \mspace{-6mu}&\mspace{-6mu} = \mspace{-6mu}&\mspace{-6mu} \frac{5}{68} \mspace{-6mu}&\mspace{-6mu} + \mspace{-6mu}&\mspace{-6mu} \frac{19}{204} x_{5} \mspace{-6mu}&\mspace{-6mu} - \mspace{-6mu}&\mspace{-6mu} \frac{1}{51} x_{6} \mspace{-6mu}&\mspace{-6mu} - \mspace{-6mu}&\mspace{-6mu} \frac{37}{51} x_{1} \mspace{-6mu}&\mspace{-6mu} - \mspace{-6mu}&\mspace{-6mu} \frac{5}{68} x_{0} \mspace{-6mu}&\mspace{-6mu} - \mspace{-6mu}&\mspace{-6mu} \frac{247}{204} x_{4}\\
x_{7} \mspace{-6mu}&\mspace{-6mu} = \mspace{-6mu}&\mspace{-6mu} \frac{9}{68} \mspace{-6mu}&\mspace{-6mu} + \mspace{-6mu}&\mspace{-6mu} \frac{143}{204} x_{5} \mspace{-6mu}&\mspace{-6mu} + \mspace{-6mu}&\mspace{-6mu} \frac{22}{51} x_{6} \mspace{-6mu}&\mspace{-6mu} + \mspace{-6mu}&\mspace{-6mu} \frac{559}{51} x_{1} \mspace{-6mu}&\mspace{-6mu} - \mspace{-6mu}&\mspace{-6mu} \frac{9}{68} x_{0} \mspace{-6mu}&\mspace{-6mu} - \mspace{-6mu}&\mspace{-6mu} \frac{2879}{204} x_{4}\\
x_{8} \mspace{-6mu}&\mspace{-6mu} = \mspace{-6mu}&\mspace{-6mu} \frac{3}{17} \mspace{-6mu}&\mspace{-6mu} + \mspace{-6mu}&\mspace{-6mu} \frac{25}{51} x_{5} \mspace{-6mu}&\mspace{-6mu} + \mspace{-6mu}&\mspace{-6mu} \frac{35}{51} x_{6} \mspace{-6mu}&\mspace{-6mu} + \mspace{-6mu}&\mspace{-6mu} \frac{173}{51} x_{1} \mspace{-6mu}&\mspace{-6mu} - \mspace{-6mu}&\mspace{-6mu} \frac{3}{17} x_{0} \mspace{-6mu}&\mspace{-6mu} - \mspace{-6mu}&\mspace{-6mu} \frac{223}{51} x_{4}\\
x_{9} \mspace{-6mu}&\mspace{-6mu} = \mspace{-6mu}&\mspace{-6mu} \frac{4}{17} \mspace{-6mu}&\mspace{-6mu} + \mspace{-6mu}&\mspace{-6mu} \frac{73}{51} x_{5} \mspace{-6mu}&\mspace{-6mu} - \mspace{-6mu}&\mspace{-6mu} \frac{10}{51} x_{6} \mspace{-6mu}&\mspace{-6mu} - \mspace{-6mu}&\mspace{-6mu} \frac{523}{51} x_{1} \mspace{-6mu}&\mspace{-6mu} - \mspace{-6mu}&\mspace{-6mu} \frac{4}{17} x_{0} \mspace{-6mu}&\mspace{-6mu} - \mspace{-6mu}&\mspace{-6mu} \frac{235}{51} x_{4}\\
\hline
w \mspace{-6mu}&\mspace{-6mu} = \mspace{-6mu}&\mspace{-6mu} 0 \mspace{-6mu}&\mspace{-6mu}  \mspace{-6mu}&\mspace{-6mu}  \mspace{-6mu}&\mspace{-6mu}  \mspace{-6mu}&\mspace{-6mu}  \mspace{-6mu}&\mspace{-6mu}  \mspace{-6mu}&\mspace{-6mu}  \mspace{-6mu}&\mspace{-6mu} - \mspace{-6mu}&\mspace{-6mu} x_{0} \mspace{-6mu}&\mspace{-6mu}  \mspace{-6mu}&\mspace{-6mu} \\
\hline
\end{array}
\end{equation*}
Back to the original problem.
\begin{equation*}
\renewcommand{\arraystretch}{1.5} %notruncate
\begin{array}{|rcrcrcrcrcr|}
\hline
x_{3} \mspace{-6mu}&\mspace{-6mu} = \mspace{-6mu}&\mspace{-6mu} \frac{1}{34} \mspace{-6mu}&\mspace{-6mu} - \mspace{-6mu}&\mspace{-6mu} \frac{1}{34} x_{5} \mspace{-6mu}&\mspace{-6mu} + \mspace{-6mu}&\mspace{-6mu} \frac{1}{17} x_{6} \mspace{-6mu}&\mspace{-6mu} + \mspace{-6mu}&\mspace{-6mu} \frac{3}{17} x_{1} \mspace{-6mu}&\mspace{-6mu} - \mspace{-6mu}&\color{green}\mspace{-6mu} \frac{21}{34} x_{4}\\
x_{2} \mspace{-6mu}&\mspace{-6mu} = \mspace{-6mu}&\mspace{-6mu} \frac{5}{68} \mspace{-6mu}&\mspace{-6mu} + \mspace{-6mu}&\mspace{-6mu} \frac{19}{204} x_{5} \mspace{-6mu}&\mspace{-6mu} - \mspace{-6mu}&\mspace{-6mu} \frac{1}{51} x_{6} \mspace{-6mu}&\mspace{-6mu} - \mspace{-6mu}&\mspace{-6mu} \frac{37}{51} x_{1} \mspace{-6mu}&\mspace{-6mu} - \mspace{-6mu}&\color{green}\mspace{-6mu} \frac{247}{204} x_{4}\\
\color{red}x_{7} \mspace{-6mu}&\color{red}\mspace{-6mu} = \mspace{-6mu}&\color{red}\mspace{-6mu} \frac{9}{68} \mspace{-6mu}&\color{red}\mspace{-6mu} + \mspace{-6mu}&\color{red}\mspace{-6mu} \frac{143}{204} x_{5} \mspace{-6mu}&\color{red}\mspace{-6mu} + \mspace{-6mu}&\color{red}\mspace{-6mu} \frac{22}{51} x_{6} \mspace{-6mu}&\color{red}\mspace{-6mu} + \mspace{-6mu}&\color{red}\mspace{-6mu} \frac{559}{51} x_{1} \mspace{-6mu}&\color{red}\mspace{-6mu} - \mspace{-6mu}&\color{blue}\mspace{-6mu} \frac{2879}{204} x_{4}\\
x_{8} \mspace{-6mu}&\mspace{-6mu} = \mspace{-6mu}&\mspace{-6mu} \frac{3}{17} \mspace{-6mu}&\mspace{-6mu} + \mspace{-6mu}&\mspace{-6mu} \frac{25}{51} x_{5} \mspace{-6mu}&\mspace{-6mu} + \mspace{-6mu}&\mspace{-6mu} \frac{35}{51} x_{6} \mspace{-6mu}&\mspace{-6mu} + \mspace{-6mu}&\mspace{-6mu} \frac{173}{51} x_{1} \mspace{-6mu}&\mspace{-6mu} - \mspace{-6mu}&\color{green}\mspace{-6mu} \frac{223}{51} x_{4}\\
x_{9} \mspace{-6mu}&\mspace{-6mu} = \mspace{-6mu}&\mspace{-6mu} \frac{4}{17} \mspace{-6mu}&\mspace{-6mu} + \mspace{-6mu}&\mspace{-6mu} \frac{73}{51} x_{5} \mspace{-6mu}&\mspace{-6mu} - \mspace{-6mu}&\mspace{-6mu} \frac{10}{51} x_{6} \mspace{-6mu}&\mspace{-6mu} - \mspace{-6mu}&\mspace{-6mu} \frac{523}{51} x_{1} \mspace{-6mu}&\mspace{-6mu} - \mspace{-6mu}&\color{green}\mspace{-6mu} \frac{235}{51} x_{4}\\
\hline
-z \mspace{-6mu}&\mspace{-6mu} = \mspace{-6mu}&\mspace{-6mu} -\frac{7}{68} \mspace{-6mu}&\mspace{-6mu} - \mspace{-6mu}&\mspace{-6mu} \frac{13}{204} x_{5} \mspace{-6mu}&\mspace{-6mu} - \mspace{-6mu}&\mspace{-6mu} \frac{2}{51} x_{6} \mspace{-6mu}&\mspace{-6mu} - \mspace{-6mu}&\mspace{-6mu} \frac{23}{51} x_{1} \mspace{-6mu}&\mspace{-6mu} + \mspace{-6mu}&\color{green}\mspace{-6mu} \frac{169}{204} x_{4}\\
\hline
\end{array}
\end{equation*}
Entering: $x_{4}$. Leaving: $x_{7}$. 
\begin{equation*}
\renewcommand{\arraystretch}{1.5} %notruncate
\begin{array}{|rcrcrcrcrcr|}
\hline
x_{3} \mspace{-6mu}&\mspace{-6mu} = \mspace{-6mu}&\mspace{-6mu} \frac{68}{2879} \mspace{-6mu}&\mspace{-6mu} - \mspace{-6mu}&\mspace{-6mu} \frac{173}{2879} x_{5} \mspace{-6mu}&\mspace{-6mu} + \mspace{-6mu}&\mspace{-6mu} \frac{115}{2879} x_{6} \mspace{-6mu}&\mspace{-6mu} - \mspace{-6mu}&\color{green}\mspace{-6mu} \frac{873}{2879} x_{1} \mspace{-6mu}&\mspace{-6mu} + \mspace{-6mu}&\mspace{-6mu} \frac{126}{2879} x_{7}\\
x_{2} \mspace{-6mu}&\mspace{-6mu} = \mspace{-6mu}&\mspace{-6mu} \frac{179}{2879} \mspace{-6mu}&\mspace{-6mu} + \mspace{-6mu}&\mspace{-6mu} \frac{95}{2879} x_{5} \mspace{-6mu}&\mspace{-6mu} - \mspace{-6mu}&\mspace{-6mu} \frac{163}{2879} x_{6} \mspace{-6mu}&\mspace{-6mu} - \mspace{-6mu}&\color{green}\mspace{-6mu} \frac{4796}{2879} x_{1} \mspace{-6mu}&\mspace{-6mu} + \mspace{-6mu}&\mspace{-6mu} \frac{247}{2879} x_{7}\\
x_{4} \mspace{-6mu}&\mspace{-6mu} = \mspace{-6mu}&\mspace{-6mu} \frac{27}{2879} \mspace{-6mu}&\mspace{-6mu} + \mspace{-6mu}&\mspace{-6mu} \frac{143}{2879} x_{5} \mspace{-6mu}&\mspace{-6mu} + \mspace{-6mu}&\mspace{-6mu} \frac{88}{2879} x_{6} \mspace{-6mu}&\mspace{-6mu} + \mspace{-6mu}&\color{green}\mspace{-6mu} \frac{2236}{2879} x_{1} \mspace{-6mu}&\mspace{-6mu} - \mspace{-6mu}&\mspace{-6mu} \frac{204}{2879} x_{7}\\
x_{8} \mspace{-6mu}&\mspace{-6mu} = \mspace{-6mu}&\mspace{-6mu} \frac{390}{2879} \mspace{-6mu}&\mspace{-6mu} + \mspace{-6mu}&\mspace{-6mu} \frac{786}{2879} x_{5} \mspace{-6mu}&\mspace{-6mu} + \mspace{-6mu}&\mspace{-6mu} \frac{1591}{2879} x_{6} \mspace{-6mu}&\mspace{-6mu} - \mspace{-6mu}&\color{green}\mspace{-6mu} \frac{11}{2879} x_{1} \mspace{-6mu}&\mspace{-6mu} + \mspace{-6mu}&\mspace{-6mu} \frac{892}{2879} x_{7}\\
\color{red}x_{9} \mspace{-6mu}&\color{red}\mspace{-6mu} = \mspace{-6mu}&\color{red}\mspace{-6mu} \frac{553}{2879} \mspace{-6mu}&\color{red}\mspace{-6mu} + \mspace{-6mu}&\color{red}\mspace{-6mu} \frac{3462}{2879} x_{5} \mspace{-6mu}&\color{red}\mspace{-6mu} - \mspace{-6mu}&\color{red}\mspace{-6mu} \frac{970}{2879} x_{6} \mspace{-6mu}&\color{red}\mspace{-6mu} - \mspace{-6mu}&\color{blue}\mspace{-6mu} \frac{39827}{2879} x_{1} \mspace{-6mu}&\color{red}\mspace{-6mu} + \mspace{-6mu}&\color{red}\mspace{-6mu} \frac{940}{2879} x_{7}\\
\hline
-z \mspace{-6mu}&\mspace{-6mu} = \mspace{-6mu}&\mspace{-6mu} -\frac{274}{2879} \mspace{-6mu}&\mspace{-6mu} - \mspace{-6mu}&\mspace{-6mu} \frac{65}{2879} x_{5} \mspace{-6mu}&\mspace{-6mu} - \mspace{-6mu}&\mspace{-6mu} \frac{40}{2879} x_{6} \mspace{-6mu}&\mspace{-6mu} + \mspace{-6mu}&\color{green}\mspace{-6mu} \frac{554}{2879} x_{1} \mspace{-6mu}&\mspace{-6mu} - \mspace{-6mu}&\mspace{-6mu} \frac{169}{2879} x_{7}\\
\hline
\end{array}
\end{equation*}
Entering: $x_{1}$. Leaving: $x_{9}$. 
\begin{equation*}
\renewcommand{\arraystretch}{1.5} %notruncate
\begin{array}{|rcrcrcrcrcr|}
\hline
x_{3} \mspace{-6mu}&\mspace{-6mu} = \mspace{-6mu}&\mspace{-6mu} \frac{773}{39827} \mspace{-6mu}&\mspace{-6mu} - \mspace{-6mu}&\mspace{-6mu} \frac{3443}{39827} x_{5} \mspace{-6mu}&\mspace{-6mu} + \mspace{-6mu}&\mspace{-6mu} \frac{1885}{39827} x_{6} \mspace{-6mu}&\mspace{-6mu} + \mspace{-6mu}&\mspace{-6mu} \frac{873}{39827} x_{9} \mspace{-6mu}&\mspace{-6mu} + \mspace{-6mu}&\mspace{-6mu} \frac{1458}{39827} x_{7}\\
x_{2} \mspace{-6mu}&\mspace{-6mu} = \mspace{-6mu}&\mspace{-6mu} \frac{1555}{39827} \mspace{-6mu}&\mspace{-6mu} - \mspace{-6mu}&\mspace{-6mu} \frac{4453}{39827} x_{5} \mspace{-6mu}&\mspace{-6mu} - \mspace{-6mu}&\mspace{-6mu} \frac{639}{39827} x_{6} \mspace{-6mu}&\mspace{-6mu} + \mspace{-6mu}&\mspace{-6mu} \frac{4796}{39827} x_{9} \mspace{-6mu}&\mspace{-6mu} + \mspace{-6mu}&\mspace{-6mu} \frac{1851}{39827} x_{7}\\
x_{4} \mspace{-6mu}&\mspace{-6mu} = \mspace{-6mu}&\mspace{-6mu} \frac{803}{39827} \mspace{-6mu}&\mspace{-6mu} + \mspace{-6mu}&\mspace{-6mu} \frac{4667}{39827} x_{5} \mspace{-6mu}&\mspace{-6mu} + \mspace{-6mu}&\mspace{-6mu} \frac{464}{39827} x_{6} \mspace{-6mu}&\mspace{-6mu} - \mspace{-6mu}&\mspace{-6mu} \frac{2236}{39827} x_{9} \mspace{-6mu}&\mspace{-6mu} - \mspace{-6mu}&\mspace{-6mu} \frac{2092}{39827} x_{7}\\
x_{8} \mspace{-6mu}&\mspace{-6mu} = \mspace{-6mu}&\mspace{-6mu} \frac{5393}{39827} \mspace{-6mu}&\mspace{-6mu} + \mspace{-6mu}&\mspace{-6mu} \frac{10860}{39827} x_{5} \mspace{-6mu}&\mspace{-6mu} + \mspace{-6mu}&\mspace{-6mu} \frac{22013}{39827} x_{6} \mspace{-6mu}&\mspace{-6mu} + \mspace{-6mu}&\mspace{-6mu} \frac{11}{39827} x_{9} \mspace{-6mu}&\mspace{-6mu} + \mspace{-6mu}&\mspace{-6mu} \frac{12336}{39827} x_{7}\\
x_{1} \mspace{-6mu}&\mspace{-6mu} = \mspace{-6mu}&\mspace{-6mu} \frac{553}{39827} \mspace{-6mu}&\mspace{-6mu} + \mspace{-6mu}&\mspace{-6mu} \frac{3462}{39827} x_{5} \mspace{-6mu}&\mspace{-6mu} - \mspace{-6mu}&\mspace{-6mu} \frac{970}{39827} x_{6} \mspace{-6mu}&\mspace{-6mu} - \mspace{-6mu}&\mspace{-6mu} \frac{2879}{39827} x_{9} \mspace{-6mu}&\mspace{-6mu} + \mspace{-6mu}&\mspace{-6mu} \frac{940}{39827} x_{7}\\
\hline
-z \mspace{-6mu}&\mspace{-6mu} = \mspace{-6mu}&\mspace{-6mu} -\frac{3684}{39827} \mspace{-6mu}&\mspace{-6mu} - \mspace{-6mu}&\mspace{-6mu} \frac{233}{39827} x_{5} \mspace{-6mu}&\mspace{-6mu} - \mspace{-6mu}&\mspace{-6mu} \frac{740}{39827} x_{6} \mspace{-6mu}&\mspace{-6mu} - \mspace{-6mu}&\mspace{-6mu} \frac{554}{39827} x_{9} \mspace{-6mu}&\mspace{-6mu} - \mspace{-6mu}&\mspace{-6mu} \frac{2157}{39827} x_{7}\\
\hline
\end{array}
\end{equation*}
The optimal value: $\frac{3684}{39827}$. An optimal solution: $\left(\frac{553}{39827},\,\frac{1555}{39827},\,\frac{773}{39827},\,\frac{803}{39827}\right)$.

    

    \begin{tcolorbox}[breakable, size=fbox, boxrule=1pt, pad at break*=1mm,colback=cellbackground, colframe=cellborder]
\prompt{In}{incolor}{4}{\boxspacing}
\begin{Verbatim}[commandchars=\\\{\}]
\PY{n}{D} \PY{o}{=} \PY{n}{P}\PY{o}{.}\PY{n}{final\PYZus{}dictionary}\PY{p}{(}\PY{p}{)}

\PY{n}{W} \PY{o}{=} \PY{o}{\PYZhy{}}\PY{n}{D}\PY{o}{.}\PY{n}{objective\PYZus{}value}\PY{p}{(}\PY{p}{)}
\PY{n}{g} \PY{o}{=} \PY{l+m+mi}{1}\PY{o}{/}\PY{n}{W}

\PY{n}{solution} \PY{o}{=} \PY{n}{D}\PY{o}{.}\PY{n}{basic\PYZus{}solution}\PY{p}{(}\PY{p}{)}

\PY{n}{x} \PY{o}{=} \PY{p}{[}\PY{n}{i} \PY{o}{*} \PY{n}{g} \PY{k}{for} \PY{n}{i} \PY{o+ow}{in} \PY{n}{solution}\PY{p}{]}
\PY{n+nb}{print}\PY{p}{(}\PY{n}{x}\PY{p}{)}
\end{Verbatim}
\end{tcolorbox}

    \begin{Verbatim}[commandchars=\\\{\}]
[553/3684, 1555/3684, 773/3684, 803/3684]
    \end{Verbatim}

    Таким образом, оптимальная смешанная стратегия игрока \(A\):
\(\left(\frac{553}{3684}, \frac{1555}{3684}, \frac{773}{3684}, \frac{803}{3684}\right)\).

    Сформулируем задачу ЛП для игрока \(B\):

    \begin{tcolorbox}[breakable, size=fbox, boxrule=1pt, pad at break*=1mm,colback=cellbackground, colframe=cellborder]
\prompt{In}{incolor}{5}{\boxspacing}
\begin{Verbatim}[commandchars=\\\{\}]
\PY{n}{B} \PY{o}{=} \PY{n}{C}
\PY{n}{b} \PY{o}{=} \PY{p}{[}\PY{l+m+mi}{1}\PY{p}{]} \PY{o}{*} \PY{n}{B}\PY{o}{.}\PY{n}{nrows}\PY{p}{(}\PY{p}{)}
\PY{n}{c} \PY{o}{=} \PY{p}{[}\PY{l+m+mi}{1}\PY{p}{]} \PY{o}{*} \PY{n}{B}\PY{o}{.}\PY{n}{ncols}\PY{p}{(}\PY{p}{)}

\PY{n}{P} \PY{o}{=} \PY{n}{InteractiveLPProblem}\PY{p}{(}\PY{n}{B}\PY{p}{,} \PY{n}{b}\PY{p}{,} \PY{n}{c}\PY{p}{,} \PY{n}{constraint\PYZus{}type}\PY{o}{=}\PY{p}{[}\PY{l+s+s2}{\PYZdq{}}\PY{l+s+s2}{\PYZlt{}=}\PY{l+s+s2}{\PYZdq{}}\PY{p}{]} \PY{o}{*} \PY{n+nb}{len}\PY{p}{(}\PY{n}{b}\PY{p}{)}\PY{p}{,} \PY{n}{problem\PYZus{}type}\PY{o}{=}\PY{l+s+s2}{\PYZdq{}}\PY{l+s+s2}{max}\PY{l+s+s2}{\PYZdq{}}\PY{p}{,} \PY{n}{variable\PYZus{}type}\PY{o}{=}\PY{l+s+s1}{\PYZsq{}}\PY{l+s+s1}{\PYZgt{}=}\PY{l+s+s1}{\PYZsq{}}\PY{p}{)}
\end{Verbatim}
\end{tcolorbox}

    \begin{tcolorbox}[breakable, size=fbox, boxrule=1pt, pad at break*=1mm,colback=cellbackground, colframe=cellborder]
\prompt{In}{incolor}{6}{\boxspacing}
\begin{Verbatim}[commandchars=\\\{\}]
\PY{n}{P} \PY{o}{=} \PY{n}{P}\PY{o}{.}\PY{n}{standard\PYZus{}form}\PY{p}{(}\PY{p}{)}
\PY{n}{Latex}\PY{p}{(}\PY{n}{P}\PY{o}{.}\PY{n}{run\PYZus{}simplex\PYZus{}method}\PY{p}{(}\PY{p}{)}\PY{p}{)}
\end{Verbatim}
\end{tcolorbox}
 
            
\prompt{Out}{outcolor}{6}{}
    
    \begin{equation*}
\renewcommand{\arraystretch}{1.5} %notruncate
\begin{array}{|rcrcrcrcrcrcr|}
\hline
x_{6} \mspace{-6mu}&\mspace{-6mu} = \mspace{-6mu}&\mspace{-6mu} 1 \mspace{-6mu}&\mspace{-6mu} - \mspace{-6mu}&\color{green}\mspace{-6mu} 8 x_{1} \mspace{-6mu}&\mspace{-6mu} - \mspace{-6mu}&\mspace{-6mu} x_{2} \mspace{-6mu}&\mspace{-6mu} - \mspace{-6mu}&\mspace{-6mu} 17 x_{3} \mspace{-6mu}&\mspace{-6mu} - \mspace{-6mu}&\mspace{-6mu} 8 x_{4} \mspace{-6mu}&\mspace{-6mu} - \mspace{-6mu}&\mspace{-6mu} x_{5}\\
x_{7} \mspace{-6mu}&\mspace{-6mu} = \mspace{-6mu}&\mspace{-6mu} 1 \mspace{-6mu}&\mspace{-6mu} - \mspace{-6mu}&\color{green}\mspace{-6mu} 12 x_{1} \mspace{-6mu}&\mspace{-6mu} - \mspace{-6mu}&\mspace{-6mu} 6 x_{2} \mspace{-6mu}&\mspace{-6mu} - \mspace{-6mu}&\mspace{-6mu} 11 x_{3} \mspace{-6mu}&\mspace{-6mu} - \mspace{-6mu}&\mspace{-6mu} 10 x_{4} \mspace{-6mu}&\mspace{-6mu} - \mspace{-6mu}&\mspace{-6mu} 16 x_{5}\\
x_{8} \mspace{-6mu}&\mspace{-6mu} = \mspace{-6mu}&\mspace{-6mu} 1 \mspace{-6mu}&\mspace{-6mu} - \mspace{-6mu}&\color{green}\mspace{-6mu} 4 x_{1} \mspace{-6mu}&\mspace{-6mu} - \mspace{-6mu}&\mspace{-6mu} 19 x_{2} \mspace{-6mu}&\mspace{-6mu} - \mspace{-6mu}&\mspace{-6mu} 11 x_{3} \mspace{-6mu}&\mspace{-6mu} - \mspace{-6mu}&\mspace{-6mu} 15 x_{4} \mspace{-6mu}&\mspace{-6mu} - \mspace{-6mu}&\mspace{-6mu} 2 x_{5}\\
\color{red}x_{9} \mspace{-6mu}&\color{red}\mspace{-6mu} = \mspace{-6mu}&\color{red}\mspace{-6mu} 1 \mspace{-6mu}&\color{red}\mspace{-6mu} - \mspace{-6mu}&\color{blue}\mspace{-6mu} 17 x_{1} \mspace{-6mu}&\color{red}\mspace{-6mu} - \mspace{-6mu}&\color{red}\mspace{-6mu} 19 x_{2} \mspace{-6mu}&\color{red}\mspace{-6mu} - \mspace{-6mu}&\color{red}\mspace{-6mu} 6 x_{3} \mspace{-6mu}&\color{red}\mspace{-6mu} - \mspace{-6mu}&\color{red}\mspace{-6mu} 17 x_{4} \mspace{-6mu}&\color{red}\mspace{-6mu} - \mspace{-6mu}&\color{red}\mspace{-6mu} 16 x_{5}\\
\hline
z \mspace{-6mu}&\mspace{-6mu} = \mspace{-6mu}&\mspace{-6mu} 0 \mspace{-6mu}&\mspace{-6mu} + \mspace{-6mu}&\color{green}\mspace{-6mu} x_{1} \mspace{-6mu}&\mspace{-6mu} + \mspace{-6mu}&\mspace{-6mu} x_{2} \mspace{-6mu}&\mspace{-6mu} + \mspace{-6mu}&\mspace{-6mu} x_{3} \mspace{-6mu}&\mspace{-6mu} + \mspace{-6mu}&\mspace{-6mu} x_{4} \mspace{-6mu}&\mspace{-6mu} + \mspace{-6mu}&\mspace{-6mu} x_{5}\\
\hline
\end{array}
\end{equation*}
Entering: $x_{1}$. Leaving: $x_{9}$. 
\begin{equation*}
\renewcommand{\arraystretch}{1.5} %notruncate
\begin{array}{|rcrcrcrcrcrcr|}
\hline
\color{red}x_{6} \mspace{-6mu}&\color{red}\mspace{-6mu} = \mspace{-6mu}&\color{red}\mspace{-6mu} \frac{9}{17} \mspace{-6mu}&\color{red}\mspace{-6mu} + \mspace{-6mu}&\color{red}\mspace{-6mu} \frac{8}{17} x_{9} \mspace{-6mu}&\color{red}\mspace{-6mu} + \mspace{-6mu}&\color{red}\mspace{-6mu} \frac{135}{17} x_{2} \mspace{-6mu}&\color{red}\mspace{-6mu} - \mspace{-6mu}&\color{blue}\mspace{-6mu} \frac{241}{17} x_{3} \mspace{-6mu}&\color{red}\mspace{-6mu}  \mspace{-6mu}&\color{red}\mspace{-6mu}  \mspace{-6mu}&\color{red}\mspace{-6mu} + \mspace{-6mu}&\color{red}\mspace{-6mu} \frac{111}{17} x_{5}\\
x_{7} \mspace{-6mu}&\mspace{-6mu} = \mspace{-6mu}&\mspace{-6mu} \frac{5}{17} \mspace{-6mu}&\mspace{-6mu} + \mspace{-6mu}&\mspace{-6mu} \frac{12}{17} x_{9} \mspace{-6mu}&\mspace{-6mu} + \mspace{-6mu}&\mspace{-6mu} \frac{126}{17} x_{2} \mspace{-6mu}&\mspace{-6mu} - \mspace{-6mu}&\color{green}\mspace{-6mu} \frac{115}{17} x_{3} \mspace{-6mu}&\mspace{-6mu} + \mspace{-6mu}&\mspace{-6mu} 2 x_{4} \mspace{-6mu}&\mspace{-6mu} - \mspace{-6mu}&\mspace{-6mu} \frac{80}{17} x_{5}\\
x_{8} \mspace{-6mu}&\mspace{-6mu} = \mspace{-6mu}&\mspace{-6mu} \frac{13}{17} \mspace{-6mu}&\mspace{-6mu} + \mspace{-6mu}&\mspace{-6mu} \frac{4}{17} x_{9} \mspace{-6mu}&\mspace{-6mu} - \mspace{-6mu}&\mspace{-6mu} \frac{247}{17} x_{2} \mspace{-6mu}&\mspace{-6mu} - \mspace{-6mu}&\color{green}\mspace{-6mu} \frac{163}{17} x_{3} \mspace{-6mu}&\mspace{-6mu} - \mspace{-6mu}&\mspace{-6mu} 11 x_{4} \mspace{-6mu}&\mspace{-6mu} + \mspace{-6mu}&\mspace{-6mu} \frac{30}{17} x_{5}\\
x_{1} \mspace{-6mu}&\mspace{-6mu} = \mspace{-6mu}&\mspace{-6mu} \frac{1}{17} \mspace{-6mu}&\mspace{-6mu} - \mspace{-6mu}&\mspace{-6mu} \frac{1}{17} x_{9} \mspace{-6mu}&\mspace{-6mu} - \mspace{-6mu}&\mspace{-6mu} \frac{19}{17} x_{2} \mspace{-6mu}&\mspace{-6mu} - \mspace{-6mu}&\color{green}\mspace{-6mu} \frac{6}{17} x_{3} \mspace{-6mu}&\mspace{-6mu} - \mspace{-6mu}&\mspace{-6mu} x_{4} \mspace{-6mu}&\mspace{-6mu} - \mspace{-6mu}&\mspace{-6mu} \frac{16}{17} x_{5}\\
\hline
z \mspace{-6mu}&\mspace{-6mu} = \mspace{-6mu}&\mspace{-6mu} \frac{1}{17} \mspace{-6mu}&\mspace{-6mu} - \mspace{-6mu}&\mspace{-6mu} \frac{1}{17} x_{9} \mspace{-6mu}&\mspace{-6mu} - \mspace{-6mu}&\mspace{-6mu} \frac{2}{17} x_{2} \mspace{-6mu}&\mspace{-6mu} + \mspace{-6mu}&\color{green}\mspace{-6mu} \frac{11}{17} x_{3} \mspace{-6mu}&\mspace{-6mu}  \mspace{-6mu}&\mspace{-6mu}  \mspace{-6mu}&\mspace{-6mu} + \mspace{-6mu}&\mspace{-6mu} \frac{1}{17} x_{5}\\
\hline
\end{array}
\end{equation*}
Entering: $x_{3}$. Leaving: $x_{6}$. 
\begin{equation*}
\renewcommand{\arraystretch}{1.5} %notruncate
\begin{array}{|rcrcrcrcrcrcr|}
\hline
x_{3} \mspace{-6mu}&\mspace{-6mu} = \mspace{-6mu}&\mspace{-6mu} \frac{9}{241} \mspace{-6mu}&\mspace{-6mu} + \mspace{-6mu}&\mspace{-6mu} \frac{8}{241} x_{9} \mspace{-6mu}&\mspace{-6mu} + \mspace{-6mu}&\color{green}\mspace{-6mu} \frac{135}{241} x_{2} \mspace{-6mu}&\mspace{-6mu} - \mspace{-6mu}&\mspace{-6mu} \frac{17}{241} x_{6} \mspace{-6mu}&\mspace{-6mu}  \mspace{-6mu}&\mspace{-6mu}  \mspace{-6mu}&\mspace{-6mu} + \mspace{-6mu}&\mspace{-6mu} \frac{111}{241} x_{5}\\
x_{7} \mspace{-6mu}&\mspace{-6mu} = \mspace{-6mu}&\mspace{-6mu} \frac{10}{241} \mspace{-6mu}&\mspace{-6mu} + \mspace{-6mu}&\mspace{-6mu} \frac{116}{241} x_{9} \mspace{-6mu}&\mspace{-6mu} + \mspace{-6mu}&\color{green}\mspace{-6mu} \frac{873}{241} x_{2} \mspace{-6mu}&\mspace{-6mu} + \mspace{-6mu}&\mspace{-6mu} \frac{115}{241} x_{6} \mspace{-6mu}&\mspace{-6mu} + \mspace{-6mu}&\mspace{-6mu} 2 x_{4} \mspace{-6mu}&\mspace{-6mu} - \mspace{-6mu}&\mspace{-6mu} \frac{1885}{241} x_{5}\\
\color{red}x_{8} \mspace{-6mu}&\color{red}\mspace{-6mu} = \mspace{-6mu}&\color{red}\mspace{-6mu} \frac{98}{241} \mspace{-6mu}&\color{red}\mspace{-6mu} - \mspace{-6mu}&\color{red}\mspace{-6mu} \frac{20}{241} x_{9} \mspace{-6mu}&\color{red}\mspace{-6mu} - \mspace{-6mu}&\color{blue}\mspace{-6mu} \frac{4796}{241} x_{2} \mspace{-6mu}&\color{red}\mspace{-6mu} + \mspace{-6mu}&\color{red}\mspace{-6mu} \frac{163}{241} x_{6} \mspace{-6mu}&\color{red}\mspace{-6mu} - \mspace{-6mu}&\color{red}\mspace{-6mu} 11 x_{4} \mspace{-6mu}&\color{red}\mspace{-6mu} - \mspace{-6mu}&\color{red}\mspace{-6mu} \frac{639}{241} x_{5}\\
x_{1} \mspace{-6mu}&\mspace{-6mu} = \mspace{-6mu}&\mspace{-6mu} \frac{11}{241} \mspace{-6mu}&\mspace{-6mu} - \mspace{-6mu}&\mspace{-6mu} \frac{17}{241} x_{9} \mspace{-6mu}&\mspace{-6mu} - \mspace{-6mu}&\color{green}\mspace{-6mu} \frac{317}{241} x_{2} \mspace{-6mu}&\mspace{-6mu} + \mspace{-6mu}&\mspace{-6mu} \frac{6}{241} x_{6} \mspace{-6mu}&\mspace{-6mu} - \mspace{-6mu}&\mspace{-6mu} x_{4} \mspace{-6mu}&\mspace{-6mu} - \mspace{-6mu}&\mspace{-6mu} \frac{266}{241} x_{5}\\
\hline
z \mspace{-6mu}&\mspace{-6mu} = \mspace{-6mu}&\mspace{-6mu} \frac{20}{241} \mspace{-6mu}&\mspace{-6mu} - \mspace{-6mu}&\mspace{-6mu} \frac{9}{241} x_{9} \mspace{-6mu}&\mspace{-6mu} + \mspace{-6mu}&\color{green}\mspace{-6mu} \frac{59}{241} x_{2} \mspace{-6mu}&\mspace{-6mu} - \mspace{-6mu}&\mspace{-6mu} \frac{11}{241} x_{6} \mspace{-6mu}&\mspace{-6mu}  \mspace{-6mu}&\mspace{-6mu}  \mspace{-6mu}&\mspace{-6mu} + \mspace{-6mu}&\mspace{-6mu} \frac{86}{241} x_{5}\\
\hline
\end{array}
\end{equation*}
Entering: $x_{2}$. Leaving: $x_{8}$. 
\begin{equation*}
\renewcommand{\arraystretch}{1.5} %notruncate
\begin{array}{|rcrcrcrcrcrcr|}
\hline
x_{3} \mspace{-6mu}&\mspace{-6mu} = \mspace{-6mu}&\mspace{-6mu} \frac{117}{2398} \mspace{-6mu}&\mspace{-6mu} + \mspace{-6mu}&\mspace{-6mu} \frac{37}{1199} x_{9} \mspace{-6mu}&\mspace{-6mu} - \mspace{-6mu}&\mspace{-6mu} \frac{135}{4796} x_{8} \mspace{-6mu}&\mspace{-6mu} - \mspace{-6mu}&\mspace{-6mu} \frac{247}{4796} x_{6} \mspace{-6mu}&\mspace{-6mu} - \mspace{-6mu}&\mspace{-6mu} \frac{135}{436} x_{4} \mspace{-6mu}&\mspace{-6mu} + \mspace{-6mu}&\color{green}\mspace{-6mu} \frac{1851}{4796} x_{5}\\
\color{red}x_{7} \mspace{-6mu}&\color{red}\mspace{-6mu} = \mspace{-6mu}&\color{red}\mspace{-6mu} \frac{277}{2398} \mspace{-6mu}&\color{red}\mspace{-6mu} + \mspace{-6mu}&\color{red}\mspace{-6mu} \frac{559}{1199} x_{9} \mspace{-6mu}&\color{red}\mspace{-6mu} - \mspace{-6mu}&\color{red}\mspace{-6mu} \frac{873}{4796} x_{8} \mspace{-6mu}&\color{red}\mspace{-6mu} + \mspace{-6mu}&\color{red}\mspace{-6mu} \frac{2879}{4796} x_{6} \mspace{-6mu}&\color{red}\mspace{-6mu} - \mspace{-6mu}&\color{red}\mspace{-6mu} \frac{1}{436} x_{4} \mspace{-6mu}&\color{red}\mspace{-6mu} - \mspace{-6mu}&\color{blue}\mspace{-6mu} \frac{39827}{4796} x_{5}\\
x_{2} \mspace{-6mu}&\mspace{-6mu} = \mspace{-6mu}&\mspace{-6mu} \frac{49}{2398} \mspace{-6mu}&\mspace{-6mu} - \mspace{-6mu}&\mspace{-6mu} \frac{5}{1199} x_{9} \mspace{-6mu}&\mspace{-6mu} - \mspace{-6mu}&\mspace{-6mu} \frac{241}{4796} x_{8} \mspace{-6mu}&\mspace{-6mu} + \mspace{-6mu}&\mspace{-6mu} \frac{163}{4796} x_{6} \mspace{-6mu}&\mspace{-6mu} - \mspace{-6mu}&\mspace{-6mu} \frac{241}{436} x_{4} \mspace{-6mu}&\mspace{-6mu} - \mspace{-6mu}&\color{green}\mspace{-6mu} \frac{639}{4796} x_{5}\\
x_{1} \mspace{-6mu}&\mspace{-6mu} = \mspace{-6mu}&\mspace{-6mu} \frac{45}{2398} \mspace{-6mu}&\mspace{-6mu} - \mspace{-6mu}&\mspace{-6mu} \frac{78}{1199} x_{9} \mspace{-6mu}&\mspace{-6mu} + \mspace{-6mu}&\mspace{-6mu} \frac{317}{4796} x_{8} \mspace{-6mu}&\mspace{-6mu} - \mspace{-6mu}&\mspace{-6mu} \frac{95}{4796} x_{6} \mspace{-6mu}&\mspace{-6mu} - \mspace{-6mu}&\mspace{-6mu} \frac{119}{436} x_{4} \mspace{-6mu}&\mspace{-6mu} - \mspace{-6mu}&\color{green}\mspace{-6mu} \frac{4453}{4796} x_{5}\\
\hline
z \mspace{-6mu}&\mspace{-6mu} = \mspace{-6mu}&\mspace{-6mu} \frac{211}{2398} \mspace{-6mu}&\mspace{-6mu} - \mspace{-6mu}&\mspace{-6mu} \frac{46}{1199} x_{9} \mspace{-6mu}&\mspace{-6mu} - \mspace{-6mu}&\mspace{-6mu} \frac{59}{4796} x_{8} \mspace{-6mu}&\mspace{-6mu} - \mspace{-6mu}&\mspace{-6mu} \frac{179}{4796} x_{6} \mspace{-6mu}&\mspace{-6mu} - \mspace{-6mu}&\mspace{-6mu} \frac{59}{436} x_{4} \mspace{-6mu}&\mspace{-6mu} + \mspace{-6mu}&\color{green}\mspace{-6mu} \frac{1555}{4796} x_{5}\\
\hline
\end{array}
\end{equation*}
Entering: $x_{5}$. Leaving: $x_{7}$. 
\begin{equation*}
\renewcommand{\arraystretch}{1.5} %notruncate
\begin{array}{|rcrcrcrcrcrcr|}
\hline
x_{3} \mspace{-6mu}&\mspace{-6mu} = \mspace{-6mu}&\mspace{-6mu} \frac{2157}{39827} \mspace{-6mu}&\mspace{-6mu} + \mspace{-6mu}&\mspace{-6mu} \frac{2092}{39827} x_{9} \mspace{-6mu}&\mspace{-6mu} - \mspace{-6mu}&\mspace{-6mu} \frac{1458}{39827} x_{8} \mspace{-6mu}&\mspace{-6mu} - \mspace{-6mu}&\mspace{-6mu} \frac{940}{39827} x_{6} \mspace{-6mu}&\mspace{-6mu} - \mspace{-6mu}&\mspace{-6mu} \frac{12336}{39827} x_{4} \mspace{-6mu}&\mspace{-6mu} - \mspace{-6mu}&\mspace{-6mu} \frac{1851}{39827} x_{7}\\
x_{5} \mspace{-6mu}&\mspace{-6mu} = \mspace{-6mu}&\mspace{-6mu} \frac{554}{39827} \mspace{-6mu}&\mspace{-6mu} + \mspace{-6mu}&\mspace{-6mu} \frac{2236}{39827} x_{9} \mspace{-6mu}&\mspace{-6mu} - \mspace{-6mu}&\mspace{-6mu} \frac{873}{39827} x_{8} \mspace{-6mu}&\mspace{-6mu} + \mspace{-6mu}&\mspace{-6mu} \frac{2879}{39827} x_{6} \mspace{-6mu}&\mspace{-6mu} - \mspace{-6mu}&\mspace{-6mu} \frac{11}{39827} x_{4} \mspace{-6mu}&\mspace{-6mu} - \mspace{-6mu}&\mspace{-6mu} \frac{4796}{39827} x_{7}\\
x_{2} \mspace{-6mu}&\mspace{-6mu} = \mspace{-6mu}&\mspace{-6mu} \frac{740}{39827} \mspace{-6mu}&\mspace{-6mu} - \mspace{-6mu}&\mspace{-6mu} \frac{464}{39827} x_{9} \mspace{-6mu}&\mspace{-6mu} - \mspace{-6mu}&\mspace{-6mu} \frac{1885}{39827} x_{8} \mspace{-6mu}&\mspace{-6mu} + \mspace{-6mu}&\mspace{-6mu} \frac{970}{39827} x_{6} \mspace{-6mu}&\mspace{-6mu} - \mspace{-6mu}&\mspace{-6mu} \frac{22013}{39827} x_{4} \mspace{-6mu}&\mspace{-6mu} + \mspace{-6mu}&\mspace{-6mu} \frac{639}{39827} x_{7}\\
x_{1} \mspace{-6mu}&\mspace{-6mu} = \mspace{-6mu}&\mspace{-6mu} \frac{233}{39827} \mspace{-6mu}&\mspace{-6mu} - \mspace{-6mu}&\mspace{-6mu} \frac{4667}{39827} x_{9} \mspace{-6mu}&\mspace{-6mu} + \mspace{-6mu}&\mspace{-6mu} \frac{3443}{39827} x_{8} \mspace{-6mu}&\mspace{-6mu} - \mspace{-6mu}&\mspace{-6mu} \frac{3462}{39827} x_{6} \mspace{-6mu}&\mspace{-6mu} - \mspace{-6mu}&\mspace{-6mu} \frac{10860}{39827} x_{4} \mspace{-6mu}&\mspace{-6mu} + \mspace{-6mu}&\mspace{-6mu} \frac{4453}{39827} x_{7}\\
\hline
z \mspace{-6mu}&\mspace{-6mu} = \mspace{-6mu}&\mspace{-6mu} \frac{3684}{39827} \mspace{-6mu}&\mspace{-6mu} - \mspace{-6mu}&\mspace{-6mu} \frac{803}{39827} x_{9} \mspace{-6mu}&\mspace{-6mu} - \mspace{-6mu}&\mspace{-6mu} \frac{773}{39827} x_{8} \mspace{-6mu}&\mspace{-6mu} - \mspace{-6mu}&\mspace{-6mu} \frac{553}{39827} x_{6} \mspace{-6mu}&\mspace{-6mu} - \mspace{-6mu}&\mspace{-6mu} \frac{5393}{39827} x_{4} \mspace{-6mu}&\mspace{-6mu} - \mspace{-6mu}&\mspace{-6mu} \frac{1555}{39827} x_{7}\\
\hline
\end{array}
\end{equation*}
The optimal value: $\frac{3684}{39827}$. An optimal solution: $\left(\frac{233}{39827},\,\frac{740}{39827},\,\frac{2157}{39827},\,0,\,\frac{554}{39827}\right)$.

    

    \begin{tcolorbox}[breakable, size=fbox, boxrule=1pt, pad at break*=1mm,colback=cellbackground, colframe=cellborder]
\prompt{In}{incolor}{7}{\boxspacing}
\begin{Verbatim}[commandchars=\\\{\}]
\PY{n}{D} \PY{o}{=} \PY{n}{P}\PY{o}{.}\PY{n}{final\PYZus{}dictionary}\PY{p}{(}\PY{p}{)}

\PY{n}{Z} \PY{o}{=} \PY{n}{D}\PY{o}{.}\PY{n}{objective\PYZus{}value}\PY{p}{(}\PY{p}{)}
\PY{n}{h} \PY{o}{=} \PY{l+m+mi}{1}\PY{o}{/}\PY{n}{Z}

\PY{n}{solution} \PY{o}{=} \PY{n}{D}\PY{o}{.}\PY{n}{basic\PYZus{}solution}\PY{p}{(}\PY{p}{)}

\PY{n}{v} \PY{o}{=} \PY{p}{[}\PY{n}{i} \PY{o}{*} \PY{n}{h} \PY{k}{for} \PY{n}{i} \PY{o+ow}{in} \PY{n}{solution}\PY{p}{]}
\PY{n+nb}{print}\PY{p}{(}\PY{n}{v}\PY{p}{)}
\end{Verbatim}
\end{tcolorbox}

    \begin{Verbatim}[commandchars=\\\{\}]
[233/3684, 185/921, 719/1228, 0, 277/1842]
    \end{Verbatim}

    Таким образом, оптимальная смешанная стратегия игрока \(B\):
\(\left(\frac{233}{3684}, \frac{185}{921}, \frac{719}{1228}, 0, \frac{277}{1842}\right)\).


    % Add a bibliography block to the postdoc
    
    
    
\end{document}
